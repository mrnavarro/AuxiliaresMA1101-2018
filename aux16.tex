\documentclass[letterpaper,10pt]{article}

\usepackage[activeacute,spanish]{babel}
\usepackage[left=1.8cm,top=1cm,right=1.8cm, bottom=1cm,letterpaper, includeheadfoot]{geometry}
\usepackage{framed}
\usepackage{babel}
\usepackage[utf8]{inputenc}
\usepackage{algorithmic}
\usepackage{algorithm}
%\usepackage{enumitem}
\usepackage{enumerate}
\usepackage{multicol}
\usepackage{amssymb, amsmath, amsthm}
\usepackage{subcaption}
\usepackage{graphicx,txfonts}
\usepackage{lmodern,url}
\usepackage{graphicx}
\usepackage{wrapfig}
\usepackage{hyperref}
\usepackage[dvipsnames]{xcolor}
\usepackage{epigraph}
\usepackage{color}
\usepackage{cancel}
\usepackage{tikz}
\def\checkmark{\tikz\fill[scale=0.4](0,.35) -- (.25,0) -- (1,.7) -- (.25,.15) -- cycle;} 
\floatname{algorithm}{Algoritmo}

\makeatletter


\setlength\epigraphwidth{8cm}
\setlength\epigraphrule{0pt}
\usepackage{fancyhdr}
\setlength{\headheight}{15pt} 
\pagestyle{fancy}
\fancypagestyle{plain}{%
    \fancyhf{}
    \lhead{\footnotesize\itshape\bfseries\rightmark}
    \rhead{\footnotesize\itshape\bfseries\leftmark}
    }

\setlength{\parindent}{1cm}
\newenvironment{chapquote}[2][2em]
  {\setlength{\@tempdima}{#1}%
   \def\chapquote@author{#2}%
   \parshape 1 \@tempdima \dimexpr\textwidth-2\@tempdima\relax%
   \itshape}
  {\par\normalfont\hfill--\ \chapquote@author\hspace*{\@tempdima}\par\bigskip}
\makeatother

% macros
\newcommand{\heart}{\ensuremath\heartsuit}
\newcommand{\grad}{\hspace{-2mm}$\phantom{a}^{\circ}$}
\newcommand{\Q}{\mathbb Q}
\newcommand{\R}{\mathbb R}
\newcommand{\N}{\mathbb N}
\newcommand{\K}{\mathbb K}
\newcommand{\Z}{\mathbb Z}
\newcommand{\re}{\mathbb R \text{e}}
\newcommand{\im}{\mathbb I \text{m}}
\newcommand{\C}{\mathbb C}
\newcommand{\U}{\mathcal U}
\DeclareMathOperator{\gr}{gr}
\newcommand{\ssi}{\Longleftrightarrow} %si y solo si
\newcommand{\To}{\Rightarrow}      %implica
\newcommand{\tq}{\mid }            % tal que
\newcommand{\exclusivo}{\veebar }  % o exclusivo
\renewcommand{\vec}[2]{\left(\begin{array}{c}{#1}\\{#2}\end{array}\right)}
\newcommand{\texii}[2]{\begin{minipage}{0.5\textwidth} #1 \end{minipage}  
                     \begin{minipage}{0.5\textwidth} #2 \end{minipage}}

%%%operadores matematicos
\providecommand{\abs}[1]{\lvert#1 \rvert}
\providecommand{\pin}[2]{\left< #1,#2 \right>} %producto interno
\providecommand{\dpartial}[2]{\frac{\partial #1}{\partial #2}} %derivada parcial


%Teoremas, Lemas, etc.
\theoremstyle{plain}
\newtheorem{teo}{Teorema}
\newtheorem{lem}{Lema}
\newtheorem{prop}{Proposici\'on}
\newtheorem{cor}{Corolario}
\newtheorem{prob}{Problema Controlable}
\newtheorem{nota}{Notaci\'on}
\newtheorem{obs}{Observaci\'on}
\newcommand{\cupdot}{\mathbin{\mathaccent\cdot\cup}}
%%%%%%% inicio documento %%%%%%%
\begin{document}

%============Encabezado estandar============
\newpage
\pagestyle{fancy}
\fancyhf{}
\fancyhead[L]{\textit{Facultad de Ciencias Físicas y Matemáticas}}
\fancyhead[R]{\textit{Universidad de Chile}}

\begin{wrapfigure}{R}{0.2\textwidth} %this figure will be at the right
    \vspace{-5mm}
    \includegraphics[width=0.2\textwidth]{img/vaquero.png}
\end{wrapfigure}


\noindent
\textbf{MA1101-1 Introducción al Álgebra}\\
\textbf{Profesor: }Leonardo Sánchez C.\\
\textbf{Auxiliar: }Marcelo Navarro

\begin{center}
{\bf \Large Auxiliar 16: Raices de la unidad y Polinomios}\\
{\today}
\end{center}

\begin{framed}
    \begin{multicols}{2}
    \begin{itemize}
        \item Si $e^{i\theta}=e^{i\varphi}$, entonces existe algún $k \in \Z$ tal que $\theta = \varphi +2k \pi$.
        \item Sea $z \in \C$ y $n \geq 2$. Diremos que $z$ es una raiz $n$-ésima de la unidad si $z^n=1$.
        \item Las raices $n$-ésimas de la unidad son de la forma $e^{i \frac{2k\pi}{n}}$ con $k \in \{0, \dotso ,n-1 \}$.
        \item Sea $z,w \in \C$ y $n \geq 2$. Diremos que $z$ es una raiz $n$-ésima de $w$ si $w^n=1$.
        \item Las raices $n$-ésimas de $z=re^{i\theta} \in \C$ son de la forma $\sqrt[n]{r}e^{i\frac{\theta +2k\pi}{n}}$ con $k \in \{0, \dotso ,n-1 \}$.
        \item Sea $n \geq 2$. La suma de las $n$ raíces $n$-ésimas de la unidad vale 0. 
        \item Consideraremos a $(\K,+,\cdot)$ como el cuerpo $\R $ o $\C$. Un polinomio es una función $p :\K \to \K$ de la forma:
			$$
			p(x)=\sum_{k=0}^{n}a_kx^k=a_0+a_1x+ \dotso + a_nx^n
			$$
			donde $a_0,a_1,\dotso , a_n$ son constantes en $\K$ a las que llamaremos coeficientes.
		\item Al conjunto de polinomios con coeficientes en $\K$ se le denota $\K[x]$. 
		\item Sean $p,q \in \K[x]$.
			$$
			p=q \iff \text{Los coeficientes de $p$ y $q$ son iguales.}
			$$
		\item Sea $p\in \K[x]$. Definimos $\gr(p)$ (el grado) como el $k$ más grande tal que $a_k\neq 0$. Si $p \equiv 0$, diremos que $\gr(p)=-\infty$.
		\item Diremos que $p\in \K[x]$ es mónico si el coeficiente asociado a $x^{\gr(p)}$ es 1.
		\item Sea $p(x)= \sum\limits_{k=0}^{n} a_k x^k$ y $q(x)=\sum\limits_{k=0}^{n} b_k x^k$ definimos:
			\begin{enumerate}
				\item $(p+q)(x)=\sum\limits_{k=0}^{n}(a_k+b_k)x^k$
				\item $(pq)(x)=\sum\limits_{k=0}^{2n}\left(\sum\limits_{i=0}^{k}a_{i}b_{k-i}\right) x^k$
			\end{enumerate}
			Estas operaciones verifican:
		\begin{enumerate}
			\item $\gr(p+q) \leq \max \{ \gr(p), \gr(q) \}$.\\
			\item $\gr(pq ) = \gr(p)+ \gr(q)$.
		\end{enumerate}
		\item $(\K[x],+,\cdot)$ es un anillo conmutativo con unidad, que no posee divisores del 0.
		\item En $(\K[x],+,\cdot)$, los únicos elementos con inverso para $\cdot$ son los polinomios de grado 0.	
		\item Sean $p,d \in \K[x]$ con $d \neq 0$. Entonces existe un único par $q,r \in \K[x]$ tal que:
		\begin{enumerate}
			\item $p=qd+r$.
			\item $\gr(r)<\gr(d)$.
		\end{enumerate}
		A $r$ lo llamaremos resto.
		Si $r\equiv 0$, diremos que $d$ divide a $p$ y lo denotaremos por $d|p$.	
		\item \textbf{Teorema del Resto:} Sea $p \in \K[x]$ y $c \in \K$. El resto de dividir $p$ por el polinomio $(x-c)$ es $p(c)$.
		\item Diremos que $c \in \K$ es una raíz de $p \in \K[x]$ si $p(c)=0$.
		\item Si $c_1,c_2, \dotso, c_k$ son raíces distintas de $p$, entonces:
		$$
		(x-c_1)(x-c_2)\dotso (x-c_k) | p(x)
		$$
		\item Sea $n \geq 1$. Si $p \in \K[x]$ es tal que $\gr(p)=n$, entonces $p $ posee a lo más $n$ raíces distintas.
		\item Sea $n \geq 1$ y $p,q \in \K[x]$ tales que $\gr (p) \leq n$ y $\gr(q ) \leq n$. Si $p$ y $q $ coinciden en $n+1$ puntos distintos, entonces son iguales.
		\item \textbf{Teorema Fundamental del Álgebra:} \\Sea $p \in \C[x]$ tal que $\gr(p)=n \geq 1$. Entonces $p$ posee al menos una raíz en $\C$.
			

		\item Sea $p \in \C[x]$ tal que $\gr(p)=n \geq 1$. Entonces existen $\alpha,c_1, \dotso c_m \in \C$ y $l_1, \dotsm l_m \in \N$ tales que:
		$$
		p(x)= \alpha (x-c_1)^{l_1} \dotso (x-c_m)^{l_m}
		$$
		\end{itemize}
		\end{multicols}
\end{framed}
	\begin{framed}
    \begin{multicols}{2}
    \begin{itemize}
		\item Sea $p \in \C[x]$ con coeficientes en $\R$ y sea $z \in \C$ una raiz de $p$. Entonces $\overline{z}$ es una raiz de $p$.
		\item Sea $p \in \R[x]$, tal que $\gr(p)=n\geq 1$. Entonces existen valores $\alpha, c_1, \dotso,c_m, a_1,b_1, \dotso , a_s,b_s \in \R$ tales que:\\
		$
		p\equiv\alpha (x-c_1)\dotso (x-c_m)(x^2+a_1x+b_1)\dotso (x^2
		+a_sx+b_s)
		$
		\item \textbf{Teorema de la Raíz Racional:} Sea $p \in \R[x]$, con coeficientes en $\Z$. Si $r$ y $s$ son primos relativos tal que $\frac{r}{s}$ es una raíz de $p$, entonces:
			$$
			r|a_0 \hspace{2 mm} \land \hspace{2 mm} s|a_n
			$$
		\item Sea $p \in \R[x]$ mónico con coeficientes en $\Z$. Entonces toda raíz racional de $p$ es entera y divide a $a_0$.
		\item El algoritmo de Ruffini permite dividir un polinomio $p$ por $x-c$.        
    \end{itemize}
    \end{multicols}
\end{framed}

\begin{enumerate}[\bf P1.]
    \item \textbf{[Raices de un Complejo]}\\Encuentre las raíces cuartas del complejo $z=\dfrac{1+i\sqrt{3}}{1-i\sqrt{3}}$
    
	\item \textbf{[Solución de polinomios]}\\Demuestre que las soluciones de la ecuación $x^2 + x + 1 = 0$ son raíces cúbicas de la unidad.
	
	\textit{Hint: no calcule las raices del polinomio}
	
	\item \textbf{[Sumatoria de Raices de la Unidad]}\\ Sean $w_0,w_1, \dotso, w_{n-1}$ las raíces $n$-ésimas de la unidad ordenadas de manera usual (es decir, según argumento de manera creciente).
        \begin{enumerate}
        	\item Demuestre que:
        	$$
        	w_0 w_1 + w_1w_2 + \dotso + w_{n-2}w_{n-1} + w_{n-1}w_0=0
        	$$
        	\item Pruebe que $\forall k \in \{1,2, \dotso , n-1  \}$:
        	$$
        	\sum_{j=0}^{n-1} (w_j)^k=0
        	$$
        	\item Sea $z \in \C$ fijo. Calcule:
        	$$
        	\sum_{j=0}^{n-1} (z+w_j)^n
        	$$
        \end{enumerate}
    
    \item \textbf{[Factorización en $\R$ y $\C$]}
    \begin{enumerate} 
        \item Factorice en $\R$ y en $\C$ el polinomio $p(x)=x^4+3x^3-12x^2-13x-15$
        \item Considere el polinomio $p(x)=x^7+2x^5-x^4+x^3-2x^2-1$. Se sabe que $i$ es raiz de $p(x)$ de multiplicidad 2. Encuentre todas las raices y factorice $p(x)$ en $\R[x]$ y $\C[x]$ 
    \end{enumerate}
    
    \item \textbf{[¿Polinomios biyectivos?]} \\
    Sea $p \in \C[x]$.
    \begin{enumerate}
    	\item Demuestre que $p(x)$ es sobreyectivo, si y sólo si $\gr(p)\geq 1$. \\
    	\emph{Hint : Puede ser útil el teorema fundamental del álgebra.}
    	\item El objetivo de esta parte es probar que $p(x)$ es inyectivo, si y sólo si $\gr(p)=1$. 
    	\begin{enumerate}
    		\item[(i)] Demuestre que si $\gr(p)=1$, entonces $p(x)$ es inyectivo.
    		\item[(ii)] Demuestre que si $\gr(p)<1$, entonces $p(x)$ no es inyectivo.	
    		\item[(iii)] Sea $n>1$, $\lambda,a \in \C$. Definamos $q\in \C [x] $ como:
    		$$
    		q(x)=\lambda (x-a)^n
    		$$
    		Demuestre que $q(x)$ no es inyectivo.
    		\item[(iv)] Concluya la dirección que falta.
    	\end{enumerate}
    	\end{enumerate}
    
    \item Sea $p(x)=x^5+ax^2+b$ y $q(x)=x^3+cx+1$. Determine los valores de $a,b,c \in \C$ para que $q|p$.
	\item Sean $p, q,r \in \R[x]$ tal que $p(q-r) =q(p-r)$ y $\gr(r)\geq 0$. Demuestre que $p=q$.
    
    \item Calcule
	$
	\displaystyle\cos \left(\frac{2\pi}{5}  \right)
	$.
	
	\emph{Hint: Utilice las raíces quintas de la unidad de manera adecuada.}
	\item Sea $w\in \C$ una raiz cubica de la unidad con $w\neq 1$. Pruebe que
    $$ (1+w)^{3}+(1+w^{2})^{9}+(1+w^{3})^6=62 $$
    
    \item Sea $p$ un polinomio con coeficientes reales tal que $p \not \equiv 0$ tal que $i, 1,2,3$ son raíces de $p$. De el grado mínimo del polinomio y suponiendo que $p$ es del grado mínimo y mónico encuentrelo.
    
    
\end{enumerate}
\end{document}