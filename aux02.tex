\documentclass[letterpaper,11pt]{article}

\usepackage[activeacute,spanish]{babel}
\usepackage[left=1.5cm,top=1cm,right=1.5cm, bottom=1cm,letterpaper, includeheadfoot]{geometry}
\usepackage{framed}
\usepackage{babel}
\usepackage[utf8]{inputenc}
\usepackage{algorithmic}
\usepackage{algorithm}
%\usepackage{enumitem}
\usepackage{enumerate}
\usepackage{multicol}
\usepackage{amssymb, amsmath, amsthm}
\usepackage{subcaption}
\usepackage{graphicx}
\usepackage{lmodern,url}
\usepackage{graphicx}
\usepackage{wrapfig}
\usepackage{hyperref}
\usepackage[dvipsnames]{xcolor}
\usepackage{epigraph}
\usepackage{color}
\usepackage{cancel}
\usepackage{tikz}
\def\checkmark{\tikz\fill[scale=0.4](0,.35) -- (.25,0) -- (1,.7) -- (.25,.15) -- cycle;} 
\floatname{algorithm}{Algoritmo}

\makeatletter


\setlength\epigraphwidth{8cm}
\setlength\epigraphrule{0pt}
\usepackage{fancyhdr}
\setlength{\headheight}{15pt} 
\pagestyle{fancy}
\fancypagestyle{plain}{%
    \fancyhf{}
    \lhead{\footnotesize\itshape\bfseries\rightmark}
    \rhead{\footnotesize\itshape\bfseries\leftmark}
    }

\setlength{\parindent}{1cm}
\newenvironment{chapquote}[2][2em]
  {\setlength{\@tempdima}{#1}%
   \def\chapquote@author{#2}%
   \parshape 1 \@tempdima \dimexpr\textwidth-2\@tempdima\relax%
   \itshape}
  {\par\normalfont\hfill--\ \chapquote@author\hspace*{\@tempdima}\par\bigskip}
\makeatother

% macros
\newcommand{\grad}{\hspace{-2mm}$\phantom{a}^{\circ}$}
\newcommand{\Q}{\mathbb Q}
\newcommand{\R}{\mathbb R}
\newcommand{\N}{\mathbb N}
\newcommand{\Z}{\mathbb Z}
\newcommand{\C}{\mathbb C}
\newcommand{\U}{\mathcal U}
\newcommand{\ssi}{\Longleftrightarrow} %si y solo si
\newcommand{\To}{\Rightarrow}      %implica
\newcommand{\tq}{\mid }            % tal que
\newcommand{\exclusivo}{\veebar }  % o exclusivo
\renewcommand{\vec}[2]{\left(\begin{array}{c}{#1}\\{#2}\end{array}\right)}
\newcommand{\texii}[2]{\begin{minipage}{0.5\textwidth} #1 \end{minipage}  
                     \begin{minipage}{0.5\textwidth} #2 \end{minipage}}

%%%operadores matematicos
\providecommand{\abs}[1]{\lvert#1 \rvert}
\providecommand{\pin}[2]{\left< #1,#2 \right>} %producto interno
\providecommand{\dpartial}[2]{\frac{\partial #1}{\partial #2}} %derivada parcial


%Teoremas, Lemas, etc.
\theoremstyle{plain}
\newtheorem{teo}{Teorema}
\newtheorem{lem}{Lema}
\newtheorem{prop}{Proposici\'on}
\newtheorem{cor}{Corolario}
\newtheorem{prob}{Problema Controlable}
\newtheorem{nota}{Notaci\'on}
\newtheorem{obs}{Observaci\'on}

%%%%%%% inicio documento %%%%%%%
\begin{document}

%============Encabezado estandar============
\newpage
\pagestyle{fancy}
\fancyhf{}
\fancyhead[L]{\textit{Facultad de Ciencias Físicas y Matemáticas}}
\fancyhead[R]{\textit{Universidad de Chile}}

\begin{wrapfigure}{R}{0.2\textwidth} %this figure will be at the right
    \vspace{-5mm}
    \includegraphics[width=0.2\textwidth]{img/fcfm2.png}
\end{wrapfigure}


\noindent
\textbf{MA1101-1 Introducción al Álgebra}\\
\textbf{Profesor: }Leonardo Sánchez C.\\
\textbf{Auxiliar: }Marcelo Navarro

\begin{center}
{\bf \Large Auxiliar 2: Inducción}\\
{29 de Marzo de 2018}
\end{center}

\begin{framed}
			\begin{itemize}
				\item Consideremos una proposición:
$$
(\forall n\in \N , n \geq n_0 )p(n)
$$
Entonces esto es equivalente a:
$$
p(n_0) \land [(\forall n\geq n_0) p(n) \implies p(n+1)]
$$
y también es equivalente a:
$$
p(n_0) \land [(\forall n\geq n_0) (p(n_0) \land \dots \land p(n-1) )\implies p(n)]
$$
			\end{itemize}	
\end{framed}

\begin{enumerate}[\bf P1.]   
    \item \textbf{[Demostraciones con Cuantificadores]}
    \begin{enumerate}
        \item Demuestre que $\exists m \in \Z, ~ \dfrac{m-7}{2m+4}=5$
        \item Demuestre que $\forall n \in \N, [n ~\text{es par} \Longleftrightarrow n^{2} ~\text{es par}]$
    \end{enumerate}
    
    \item \textbf{[Divisibilidad]}
        \begin{enumerate}
            \item Demuestre $\forall n \in \N$
                \begin{enumerate}
                    \item $n^3+5n ~ \text{es multiplo de } 6$
                    \item $13 ~ | ~ 4^{2n+1}+3^{n+2}$
                \end{enumerate}
                \textbf{Obs: }La expresión $a ~|~ b$ se lee ``a divide a b"
        \end{enumerate}

    \item \textbf{[Pensamiento Recursivo]}\\ Sea $T_{n}$ una sucesión, definida recursivamente por:
        \begin{align*}
            T_{n}&=5T_{n-1}-4T_{n-2}, ~ \forall n \geq 3\\
            T_{1}&=3\\
            T_{2}&=15
        \end{align*}

        \begin{enumerate}
            \item Calcule $T_{3}, T_{4}$ y $T_{5}$
            \item Con estos valores, conjeture un valor no recursivo para $T_{n}$ y demuestre usando inducción que su conjetura es correcta.
        \end{enumerate}
        
    \item \textbf{[Desigualdades]}
        \begin{enumerate}
            \item Determine el menor $n_{0}\in \N$ a partir del cual es válida la desigualdad $3n + 2 < 2^{n}$ y demuestre que la desigualdad es cierta $\forall n \geq n_{0}$ usando inducción.
            \item Considere la recursión $a_1=0$, $a_n=\dfrac{1+a_{n-1}}{2+a_{n-1}}$ para $n>1$.
            
            Pruebe que $a_{n-1}<a_n$ para todo $n>1$
        \end{enumerate}
    
    \item \textbf{[Empanadas]}\\
    El responsable de marketing de una empresa que produce empanadas sugiere que estas se vendan en paquetes de 3 y 5 empanadas. El dueño se niega tajantemente a la petición argumentando que con esa restricción no las podrán vender en cantidades arbitrarias. El caballero del empaque dice que ese problema no es real, puesto que cualquier cantidad de empanadas superior o igual a 8, se puede conseguir usando solamente los paquetes de 3 y 5. ¿Quién tienen razón?, demuestrelo.
    
    \item \textbf{[Si alcanzamos... Sumatorias]}\\
        Demuestre que $2^0+2^1+2^2+...+2^n=2^{n+1}-1$ para todo $n$.
    
\end{enumerate}


\end{document}