\documentclass[letterpaper,11pt]{article}

\usepackage[activeacute,spanish]{babel}
\usepackage[left=1.8cm,top=1cm,right=1.8cm, bottom=1cm,letterpaper, includeheadfoot]{geometry}
\usepackage{framed}
\usepackage{babel}
\usepackage[utf8]{inputenc}
\usepackage{algorithmic}
\usepackage{algorithm}
%\usepackage{enumitem}
\usepackage{enumerate}
\usepackage{multicol}
\usepackage{amssymb, amsmath, amsthm}
\usepackage{subcaption}
\usepackage{graphicx,txfonts}
\usepackage{lmodern,url}
\usepackage{graphicx}
\usepackage{wrapfig}
\usepackage{hyperref}
\usepackage[dvipsnames]{xcolor}
\usepackage{epigraph}
\usepackage{color}
\usepackage{cancel}
\usepackage{tikz}
\def\checkmark{\tikz\fill[scale=0.4](0,.35) -- (.25,0) -- (1,.7) -- (.25,.15) -- cycle;} 
\floatname{algorithm}{Algoritmo}

\makeatletter


\setlength\epigraphwidth{8cm}
\setlength\epigraphrule{0pt}
\usepackage{fancyhdr}
\setlength{\headheight}{15pt} 
\pagestyle{fancy}
\fancypagestyle{plain}{%
    \fancyhf{}
    \lhead{\footnotesize\itshape\bfseries\rightmark}
    \rhead{\footnotesize\itshape\bfseries\leftmark}
    }

\setlength{\parindent}{1cm}
\newenvironment{chapquote}[2][2em]
  {\setlength{\@tempdima}{#1}%
   \def\chapquote@author{#2}%
   \parshape 1 \@tempdima \dimexpr\textwidth-2\@tempdima\relax%
   \itshape}
  {\par\normalfont\hfill--\ \chapquote@author\hspace*{\@tempdima}\par\bigskip}
\makeatother

% macros
\newcommand{\cur}[1]{\mathcal{#1}}
\newcommand{\id}{\operatorname{Id}}
\newcommand{\heart}{\ensuremath\heartsuit}
\newcommand{\grad}{\hspace{-2mm}$\phantom{a}^{\circ}$}
\newcommand{\Q}{\mathbb Q}
\newcommand{\R}{\mathbb R}
\newcommand{\N}{\mathbb N}
\newcommand{\Z}{\mathbb Z}
\newcommand{\C}{\mathbb C}
\newcommand{\U}{\mathcal U}
\newcommand{\ssi}{\Longleftrightarrow} %si y solo si
\newcommand{\To}{\Rightarrow}      %implica
\newcommand{\tq}{\mid }            % tal que
\newcommand{\exclusivo}{\veebar }  % o exclusivo
\renewcommand{\vec}[2]{\left(\begin{array}{c}{#1}\\{#2}\end{array}\right)}
\newcommand{\texii}[2]{\begin{minipage}{0.5\textwidth} #1 \end{minipage}  
                     \begin{minipage}{0.5\textwidth} #2 \end{minipage}}

%%%operadores matematicos
\providecommand{\abs}[1]{\lvert#1 \rvert}
\providecommand{\pin}[2]{\left< #1,#2 \right>} %producto interno
\providecommand{\dpartial}[2]{\frac{\partial #1}{\partial #2}} %derivada parcial


%Teoremas, Lemas, etc.
\theoremstyle{plain}
\newtheorem{teo}{Teorema}
\newtheorem{lem}{Lema}
\newtheorem{prop}{Proposici\'on}
\newtheorem{cor}{Corolario}
\newtheorem{prob}{Problema Controlable}
\newtheorem{nota}{Notaci\'on}
\newtheorem{obs}{Observaci\'on}

%%%%%%% inicio documento %%%%%%%
\begin{document}

%============Encabezado estandar============
\newpage
\pagestyle{fancy}
\fancyhf{}
\fancyhead[L]{\textit{Facultad de Ciencias Físicas y Matemáticas}}
\fancyhead[R]{\textit{Universidad de Chile}}

\begin{wrapfigure}{R}{0.2\textwidth} %this figure will be at the right
    \vspace{-5mm}
    \includegraphics[width=0.2\textwidth]{img/fcfm2.png}
\end{wrapfigure}


\noindent
\textbf{MA1101-1 Introducción al Álgebra}\\
\textbf{Profesor: }Leonardo Sánchez C.\\
\textbf{Auxiliar: }Marcelo Navarro

\begin{center}
{\bf \Large Auxiliar 10: Control 4}\\
{05 de Julio de 2018}
\end{center}

\begin{enumerate}[\bf P1.]
    \item Si $\displaystyle \sum_{k=1}^n a_k=\frac{1}{3}(n^2+5n)$ se pide determinar $a_n$
    \item Sea $n\in \N$ positivo. Calcule
        $$\displaystyle \sum_{k=1}^{2n} \frac{1}{(3+(-1)^k)^{k}}  $$
    \item Calcular $\displaystyle \sum_{k=1}^{n}\sum_{j=1}^{k} \frac{n+k}{(n+j-1)(n+j)}$
    \item \textbf{(P3 Control 2, Año 1998)}\\
Sea $A$ un conjunto no vacío y $f:A\longrightarrow A$ una función biyectiva. Denotaremos por $f^{-1}$ a la inversa de $f$. Para $n\geq 1$ definimos $f^{(n)}$ como la composición de $f$ con ella misma $n$ veces y si $n<0$ definimos $f^{(n)}=(f^{-1})^{(|n|)}$. Si $n=0$, ponemos $f^{(0)}=\id_{A}$.\\
Considere la relación en $A$ definida como:
$$x\,\cur{R}\,y \;\; \Longleftrightarrow \;\; (\exists\;n\in\Z)\; f^{(n)}(x)=y$$
\begin{enumerate}
\item Probar que $\cur{R}$ es una relación de equivalencia.
\item Considere $p\in\N\setminus \{0\}$ fijo. Si $A=\Q$ \  y \ $f:\Q\longrightarrow \Q$. Se define por $f(q)=p\cdot q$, calcular la clase de equivalencia de 0 y de 1 con respecto a $\cur{R}$.
\end{enumerate}
\end{enumerate}

\end{document}