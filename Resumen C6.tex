\documentclass[letterpaper,10pt]{article}

\usepackage[activeacute,spanish]{babel}
\usepackage[left=1.8cm,top=1cm,right=1.8cm, bottom=1cm,letterpaper, includeheadfoot]{geometry}
\usepackage{framed}
\usepackage{babel}
\usepackage[utf8]{inputenc}
\usepackage{algorithmic}
\usepackage{algorithm}
%\usepackage{enumitem}
\usepackage{enumerate}
\usepackage{multicol}
\usepackage{amssymb, amsmath, amsthm}
\usepackage{subcaption}
\usepackage{graphicx,txfonts}
\usepackage{lmodern,url}
\usepackage{graphicx}
\usepackage{wrapfig}
\usepackage{hyperref}
\usepackage[dvipsnames]{xcolor}
\usepackage{epigraph}
\usepackage{color}
\usepackage{cancel}
\usepackage{tikz}
\def\checkmark{\tikz\fill[scale=0.4](0,.35) -- (.25,0) -- (1,.7) -- (.25,.15) -- cycle;} 
\floatname{algorithm}{Algoritmo}

\makeatletter


\setlength\epigraphwidth{8cm}
\setlength\epigraphrule{0pt}
\usepackage{fancyhdr}
\setlength{\headheight}{15pt} 
\pagestyle{fancy}
\fancypagestyle{plain}{%
    \fancyhf{}
    \lhead{\footnotesize\itshape\bfseries\rightmark}
    \rhead{\footnotesize\itshape\bfseries\leftmark}
    }

\setlength{\parindent}{1cm}
\newenvironment{chapquote}[2][2em]
  {\setlength{\@tempdima}{#1}%
   \def\chapquote@author{#2}%
   \parshape 1 \@tempdima \dimexpr\textwidth-2\@tempdima\relax%
   \itshape}
  {\par\normalfont\hfill--\ \chapquote@author\hspace*{\@tempdima}\par\bigskip}
\makeatother

% macros
\newcommand{\heart}{\ensuremath\heartsuit}
\newcommand{\grad}{\hspace{-2mm}$\phantom{a}^{\circ}$}
\newcommand{\Q}{\mathbb Q}
\newcommand{\R}{\mathbb R}
\newcommand{\N}{\mathbb N}
\newcommand{\Z}{\mathbb Z}
\newcommand{\K}{\mathbb K}
\newcommand{\re}{\mathbb R \text{e}}
\newcommand{\im}{\mathbb I \text{m}}
\newcommand{\C}{\mathbb C}
\newcommand{\U}{\mathcal U}
\newcommand{\ssi}{\Longleftrightarrow} %si y solo si
\newcommand{\To}{\Rightarrow}      %implica
\newcommand{\tq}{\mid }            % tal que
\newcommand{\exclusivo}{\veebar }  % o exclusivo
\renewcommand{\vec}[2]{\left(\begin{array}{c}{#1}\\{#2}\end{array}\right)}
\newcommand{\texii}[2]{\begin{minipage}{0.5\textwidth} #1 \end{minipage}  
                     \begin{minipage}{0.5\textwidth} #2 \end{minipage}}

%%%operadores matematicos
\providecommand{\abs}[1]{\lvert#1 \rvert}
\providecommand{\pin}[2]{\left< #1,#2 \right>} %producto interno
\providecommand{\dpartial}[2]{\frac{\partial #1}{\partial #2}} %derivada parcial


%Teoremas, Lemas, etc.
\theoremstyle{plain}
\newtheorem{teo}{Teorema}
\newtheorem{lem}{Lema}
\newtheorem{prop}{Proposici\'on}
\newtheorem{cor}{Corolario}
\newtheorem{prob}{Problema Controlable}
\newtheorem{nota}{Notaci\'on}
\newtheorem{obs}{Observaci\'on}
\newcommand{\cupdot}{\mathbin{\mathaccent\cdot\cup}}
%%%%%%% inicio documento %%%%%%%
\begin{document}

%============Encabezado estandar============
\newpage
\pagestyle{fancy}
\fancyhf{}
\fancyhead[L]{\textit{Facultad de Ciencias Físicas y Matemáticas}}
\fancyhead[R]{\textit{Universidad de Chile}}

\begin{wrapfigure}{R}{0.2\textwidth} %this figure will be at the right
    \vspace{-5mm}
    \includegraphics[width=0.2\textwidth]{img/fcfm2.png}
\end{wrapfigure}


\noindent
\textbf{MA1101-1 Introducción al Álgebra}\\
\textbf{Profesor: }Leonardo Sánchez C.\\
\textbf{Auxiliar: }Marcelo Navarro

\begin{center}
{\bf \Large Resumen C6}\\
{\today}
\end{center}

\begin{framed} Estructuras Algebraicas
	\begin{multicols}{2}
	    \begin{itemize}
	        \item Para dos estructuras $(A,*),(B,\triangle)$ una función $f:A\to B$ es un \textbf{homomorfimos} de $(A,*)$ en $(B,\triangle)$ si $\forall x,y\in A, f(x*y)=f(x)\triangle f(y)$.\\
            Si $f$ es inyectiva se llamará monomorfismo\\
            Si $f$ es epiyectiva se llamará epimorfimos\\
            si $f$ es biyectiva se llamará isomorfimos\\
            Si $(A,*)=(B,\triangle)$ los homomorfimos se llamaran endomorfimos y en caso que si el endomorfimo es biyectivo se llamará automorfismo.
    
            \item Sea $f:A\to B$ un epimorfimos entre $(A,*)$ y $(B,\triangle)$, entonces se tienen las siguientes propiedadess:
                \begin{enumerate}
                    \item Si $(A,*)$ es asociativo, entonces $(B,\triangle)$ también lo es
                    \item Si $(A,*)$ es conmutativo, entonces $(B,\triangle)$ también lo es
                    \item Si $e$ es neutro de $(A,*)$, entonces $f(e)$ es neutro de $(B,\triangle)$ 
                    \item Si $a\in A$ tiene inverso $b$ para $(A,*)$, entonces $f(a)$ tiene inverso $f(b)$ para $(B,\triangle)$, es decir $f(a)^{-1}=f(a^{-1})$ 
                \end{enumerate}
                3 y 4 están sujetas a que el neutro de $(B,\triangle)$ esté en la imagen de $f$
            
            \item Sea $f$ un homomorfismo de $(A,*)$ en $(B,\triangle)$, no necesariamente epiyectivo, con neutros $e_A$ y $e_B$ respectivamente.
                \begin{enumerate}
                    \item Si $e_B \in f(A)$, entonces $e_B=f(a_A)$
                    \item Si $e_B \in f(A)$ y $a\in A$ tiene inverso $b$ para $(A,*)$, entonces $f(a)$ tiene inverso $f(b)$ para $(B,\triangle)$
                \end{enumerate}
                
            \item Sean $(A,*)$ y $(B,\triangle)$ con neutros $e_A$ y $e_B$ respectivamente y donde todos los elementos son invertibles. Un homomorfimos $f:A\to B$ de $(A,*)$ en $(B.\triangle)$ es un monomorfismo, es decir es inyectivo, si y sólo si, $f^{-1}(\{e_B\})=\{e_A\}$
            \item Sea $f:A\to B$ un homomorfismo de $(A,*)$ en $(B,\triangle)$ y $g:B \to C$ un homomorfismo de $(B,\triangle)$ en $(C,\bullet)$. Entonces la composición $g\circ f:A \to C$, es un homomorfismo de $(A,*)$ en $(C,\bullet)$
            \item Dos estructuras  $(A,*)$ y $(B,\triangle)$ son isomorfas, denotada  $(A,*) \cong (B,\triangle)$, si existe una función $f:A\to B$ que es un isomorfismo entre  $(A,*)$ y $(B,\triangle)$
            \item Si $f:A\to B$ es un isomorfismo entre $(A,*)$ y $(B,\triangle)$, entonces $f^{-1}:B\to A$ es un isomorfismo entre $(B,\triangle)$ y $(A,*)$
            \item La relación $\cong$ entre estructuras algebraicas es relación de equivalencia
	        \item Si $(G,*)$ es una estructura algebraica asociativa, con neutro y tal que todo elemento es invertible, entonces diremos que $(G,*)$ es un grupo. Si ademas la operación $*$ es conmutativa, diremos que es un grupo abeliano.
            \item Sea $(G,*)$ un grupo, entonces:
                \begin{enumerate}
                	\item El inverso de cada elemento es único.
                	\item $(\forall x \in G)$, $(x^{-1})^{-1}=x$.
                	\item $(\forall x,y \in G)$, $(x*y)^{-1}=y^{-1}*x^{-1}$.
                	\item Todo elemento $x \in G$ es cancelable.
                	\item Para todo $a,b \in G$, las ecuaciones:
                	$$
                	\begin{array}{c}
                	a*x_1=b \\x_2*a=b
                	\end{array}
                	$$ 
                	tienen solución única. Ellas son $x_1=a^{-1}*b$ y $x_2=b*a^{-1}$.
                	\item El \textbf{único} elemento idempotente de $G$ es su neutro.
                \end{enumerate}
            
	    \end{itemize}
	\end{multicols}
\end{framed}

\begin{framed}
    \begin{multicols}{2}
        \begin{itemize}
            \item Sea $(G,*)$ un grupo, y sea $H\subseteq G$. Diremos que $H$ es subgrupo de $G$ si $(H,*)$ también es grupo.
            \item \textbf{Caracterización de Subgrupo:} Sea $H \neq \emptyset$, entonces:
            $$
            (H,*) \text{ subgrupo de }(G,*)
            \Leftrightarrow
            (\forall x,y \in H)~ x*y^{-1}\in H
            $$
            \item $(\mathbf Z_n,+_n)$ es un grupo abeliano. 
            \item \textbf{Teorema de Lagrange:} Sea $(G,*)$ un grupo finito y $(H,*)$ un subgrupo de $(G,*)$. Entonces $|H|$ divide a $|G|$.
            \item Sea $(G,*)$ un grupo. A $|G|$ le llamaremos el orden del grupo.
            \item Si $f: G \to H$ es un morfismo y tanto $(G,*)$ como $(H,*)$ son grupos, entonces:
                \begin{enumerate}
                	\item $f(e_G)=e_H$.
                	\item $f(a^{-1})=f(a)^{-1}$
                \end{enumerate}
	    %%%%%%%%%%%%%%%%%%%%
	        \item A una estructura $(A,+,\cdot)$ le llamaremos anillo si satisface:
                \begin{enumerate}
                    \item $(A,+)$ es un grupo abeliano.
                    \item $\cdot$ es asociativa.
                    \item $\cdot$ distribuye con respecto a $+$.
                \end{enumerate}
            \item Sea $(A,+,\cdot)$ un anillo, tenemos la siguiente notación:
                \begin{enumerate}
                    \item Al neutro de $(A,+)$ se le suele denotar por 0, mientras que el inverso de $x$ para $+$ se denota por $-x$.
                    \item Si $(A,\cdot)$ tiene neutro a dicho neutro le llamaremos 1, más aún diremos que $(A,+,\cdot)$ es un anillo con unidad. Si $x \in A$ posee inverso para $\cdot$ lo denotaremos por $x^{-1}$.
                    \item Si $\cdot$ es conmutativa, diremos que $(A,+,\cdot)$ es un anillo conmutativo.
                \end{enumerate}
            \item $(\Z, +,\cdot)$ es un anillo conmutativo con unidad.
            \item Si $(A,+,\cdot)$ es un anillo con unidad y $|A|\geq 2$, entonces $0 \neq 1$.
            \item Sea $(A,+,\cdot)$ un anillo, entonces:
                \begin{enumerate}
                	\item 0 es absorbente.
                	\item $(\forall x,y \in A)$, $-(x \cdot y)=(-x)\cdot y = x \cdot (-y)$.
                	\item $(\forall x,y \in A)$, $(-x)\cdot (-y)= x \cdot y$.
                	\item Si $A$ tiene unidad:
                	$$
                	(\forall x \in A) \: -x=(-1)\cdot x = x  \cdot (-1)
                	$$
                \end{enumerate}
            \item $(\mathbf Z_n,+_n, \cdot_n)$ es un anillo conmutativo con unidad.
            \item Sea $(A,+,\cdot)$ un anillo. Si $x,y \in A\setminus \{0\}$ son tales que $x \cdot y=0$, diremos que $x$ e $y$ son divisores del 0.
            \item Sea $(A,+,\cdot)$ un anillo y $a \in A\setminus \{0\}$, luego:
            $$
            a \text{ es divisor del 0}\ssi a \text{ no es cancelable}
            $$
            \item Sean $(A,+,\cdot)$ y $(B,\oplus, \odot)$ dos anillos con unidad. Sea $f: A \to B$ un morfismo de anillos, es decir:
            $$	f(x+ y)=f(x)\oplus f(y) ~~~\land~~~ f(x \cdot y)=f(x ) \odot f(y)$$
            $$ \land~~~ f(1_A)=1_B  $$
            \item Sea $(\K,+, \cdot )$ un anillo conmutativo con unidad tal que todo $x \in \K \setminus \{0\}$ es invertible para $\cdot$, diremos que $(\K,+, \cdot )$ es un cuerpo.
            \item De manera equivalente $(\K+,\cdot) $ es un cuerpo si y sólo si:
                \begin{enumerate}
                    \item $(\K,+)$ es un grupo abeliano.
                    \item $(\K \setminus \{0\}, \cdot)$ es un grupo abeliano.
                    \item $\cdot $ distribuye con respecto a $+$. 
                \end{enumerate}
            \item $(\R,+,\cdot)$ y $(\Q,+,\cdot)$ son cuerpos.
            \item Un cuerpo no tiene divisores del 0.
            \item Sea $(A,+,\cdot)$ un anillo conmutativo con unidad tal que $|A|$ es finito. Entonces $(A,+,\cdot)$ no tiene divisores del cero si y sólo si $(A,+,\cdot)$ es un cuerpo.
            \item $(\mathbf{Z}_n, +, \cdot)$ es un cuerpo $\iff$ $n$ es un primo.
            \item Si $(A,+,\cdot)$ es un dominio de integridad con $|A|$ finito, entonces $(A,+,\cdot)$ es cuerpo.
        \end{itemize}
    \end{multicols}
\end{framed}

\newpage
\begin{framed} Complejos
	\begin{multicols}{2}
	    \begin{itemize}
	        \item Sea $\C=\R^2$ dotado de las siguientes operaciones, donde $z=(a,b)$ y $w=(c,d)$:
            	$$
            	\begin{array}{rcl}
            	z+w & = & (a+c, b+d) \\
            	z \cdot w & = & (ac - bd , ad+bc) 
            	\end{array}
            	$$
        	\item $(\C,+,\cdot)$ es un cuerpo
            	\begin{enumerate}
            	    \item $(0,0)$ es el neutro de $(\C,+)$
            	    \item $(1,0)$ es el neutro de $(\C,\cdot)$
            	    \item El inverso en $(\C,+)$ de $(a,b)$ es $(-a,-b)$.
            	    \item El inverso en $(\C,\cdot)$ de $(a,b)\neq(0,0)$ es $(\frac{a}{a^2+b^2},\frac{-b}{a^2+b^2})$
            	\end{enumerate} 
	
	        %\item Sea $R=\{(z_1,0) \in \C : z_1 \in \R  \}\subseteq \C$, entonces $(R,+,\cdot) \cong (\R,+,\cdot)$.
        	\item La unidad imaginaria es el complejo (0,1). Se anotará como $i$. Se cumple que $i^2=-1$
        	\item \textbf{Forma cartesiana: }La expresión $a+ib$ con $a,b\in \R$ se llama la \textit{forma cartesiana} del complejo $z=(a,b)\in \C$.
            \item Sea $z=a+bi\in \C$. Definimos la parte real y la parte imaginaria respectivamente como:
            $$
            \re(z)=a \hspace{7 mm} \im(z)=b
            $$
            
            \item $\re(\cdot)$ y $\im(\cdot)$ son morfismos de $(\C,+)$ en $(\C,+)$, es decir, son endomorfismos. Por lo tanto cumplen que $\forall z_1,z_2 \in \C$:
                \begin{enumerate}
                    \item $\re(z_1+z_2)=\re(z_1)+\re(z_2)$.
                    \item $\im(z_1+z_2)=\im(z_1)+\im(z_2)$.
                \end{enumerate}
            
            \item Sean $z,z_1,z_2 \in \C$ y $\lambda \in \R$, entonces:
                \begin{enumerate}
                    \item $\re (\lambda z)= \lambda \re (z)$.
                    \item $\im (\lambda z)= \lambda \im (z)$.
                    \item $z_1 = z_2 \Leftrightarrow [\re (z_1) = \re (z_2) \:\land\: \im (z_1) =\im (z_2)]$
                \end{enumerate}
            \item se define para $n\in \Z$
            $$i^n= \left\{ \begin{array}{lcc}
                         +1 &   si  & n\equiv_{4} 0 \\
                          +i &  si & n\equiv_{4} 1 \\
                          -1 &  si  & n\equiv_{4} 2 \\
                          -i & si & n\equiv_{4} 3
                         \end{array}
               \right.
            $$
            
            \item Dado $z=a+bi \in \C$, en virtud del teorema de pitagoras se define la distancia de $z$ hasta el origen como: $$|z|=\sqrt{a^2+b^2}$$ 
            \item Sea $z=a+bi\in \C$, se define su conjugado como $\overline{z}=a-bi$
                %\item \textbf{[Coordenadas Polares]} dado $z\in \C \setminus \{ 0\}$ su coordenada polar es $(r,\theta)\in [0,\infty) \times [0,2\pi)$ con $r=|z| $ y $\theta$ el ángulo que se forma entre el eje $OX$ y el segmento que une $z$ con el origen $O$.
            \item si $z=a+bi$. entonces se tiene la siguiente relación con su coordenada polar $(r,\theta)$
                $$ a=rcos(\theta) \land b=rsen(\theta) $$
            \item $\arg(z)$ es el ángulo de $z$ con el eje real. Se acostumbra a escoger un ángulo en el rango $(-\pi,\pi ]$.
            \item Sea $\theta \in \R$. Definimos $e^{i\theta}$ como:
            $$
            e^{i\theta}= \cos(\theta) + i \sin (\theta)
            $$
            Dado $z\in \C$ su \textbf{forma polar} es $|z|e^{iarg(z)}$
            \item Sea $\theta, \varphi \in \R$, entonces:
            	\begin{enumerate}
            		\item $|e^{i\theta}|=1$.
            		\item $\overline{e^{i\theta}}=(e^{i\theta})^{-1}=e^{-i\theta}$.
            		\item $e^{i\theta}e^{i\varphi}=e^{i(\theta + \varphi)}$.
            		\item $(e^{i\theta})^n=e^{ni\theta}=\cos(n\theta)+i \sin(n \theta) $
            	\end{enumerate}
            \item Sean $z,w \in \C$. Entonces:
                \begin{enumerate}
                    \item $\overline{\overline{z}}=z$.
                    \item $z \in \R \ssi z=\overline{z}.$
                    \item $\overline{z+w}=\overline{z}+\overline{w}$ y $\overline{z-w}=\overline{z}-\overline{w}$.
                    \item $\overline {zw} = \overline{z} \cdot \overline{w}$. Si $w\neq 0$ $\overline{\left(\frac{z}{w}\right)}=\frac{\overline{z}}{\overline{w}}$.
                    \item Si $\lambda \in \R$, entonces $\overline{\lambda z }= \lambda \overline z$.
                    
                    \item $\re(z)= \re (\overline{z})$ y $\im(z)=-\im(\overline{z})$.
                    \item $\re(z)=\frac{1}{2}(z+\overline{z})$ y $\im(z)=\frac{1}{2i}(z-\overline{z})$
                    \item Si $z \neq 0$, entonces $z^{-1}=\frac{\overline{1}}{|z|}e^{-iarg(z)}$
                    \item $|z|=|\overline{z}|$
                    \item $arg(\overline{z})=2\pi - arg(z)$
                    \item $z \cdot \overline{z}=|z|^{2}$
                    \item $|zw|=|z||w|$ y $|z+w|\leq |z|+|w|$.
                    \item Si $z \neq 0$, entonces $z^{-1}=\frac{\overline{z}}{|z|^2}$.
                    \item Si $w \neq 0$, entonces $\left| \frac{z}{w}  \right| = \frac{|z|}{|w|} $.
                \end{enumerate}
	        \item Si $e^{i\theta}=e^{i\varphi}$, entonces existe algún $k \in \Z$ tal que $\theta = \varphi +2k \pi$.
	    \end{itemize}
	\end{multicols}
\end{framed}

\end{document}