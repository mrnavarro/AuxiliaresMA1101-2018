\documentclass[letterpaper,11pt]{article}

\usepackage[activeacute,spanish]{babel}
\usepackage[left=1.8cm,top=1cm,right=1.8cm, bottom=1cm,letterpaper, includeheadfoot]{geometry}
\usepackage{framed}
\usepackage{babel}
\usepackage[utf8]{inputenc}
\usepackage{algorithmic}
\usepackage{algorithm}
%\usepackage{enumitem}
\usepackage{enumerate}
\usepackage{multicol}
\usepackage{amssymb, amsmath, amsthm}
\usepackage{subcaption}
\usepackage{graphicx,txfonts}
\usepackage{lmodern,url}
\usepackage{graphicx}
\usepackage{wrapfig}
\usepackage{hyperref}
\usepackage[dvipsnames]{xcolor}
\usepackage{epigraph}
\usepackage{color}
\usepackage{cancel}
\usepackage{tikz}
\def\checkmark{\tikz\fill[scale=0.4](0,.35) -- (.25,0) -- (1,.7) -- (.25,.15) -- cycle;} 
\floatname{algorithm}{Algoritmo}

\makeatletter




% macros
\newcommand{\heart}{\ensuremath\heartsuit}
\newcommand{\grad}{\hspace{-2mm}$\phantom{a}^{\circ}$}
\newcommand{\Q}{\mathbb Q}
\newcommand{\R}{\mathbb R}
\newcommand{\N}{\mathbb N}
\newcommand{\Z}{\mathbb Z}
\newcommand{\C}{\mathbb C}
\newcommand{\U}{\mathcal U}
\newcommand{\ssi}{\Longleftrightarrow} %si y solo si
\newcommand{\To}{\Rightarrow}      %implica
\newcommand{\tq}{\mid }            % tal que
\newcommand{\exclusivo}{\veebar }  % o exclusivo
\renewcommand{\vec}[2]{\left(\begin{array}{c}{#1}\\{#2}\end{array}\right)}
\newcommand{\texii}[2]{\begin{minipage}{0.5\textwidth} #1 \end{minipage}  
                     \begin{minipage}{0.5\textwidth} #2 \end{minipage}}

%%%operadores matematicos
\providecommand{\abs}[1]{\lvert#1 \rvert}
\providecommand{\pin}[2]{\left< #1,#2 \right>} %producto interno
\providecommand{\dpartial}[2]{\frac{\partial #1}{\partial #2}} %derivada parcial


%Teoremas, Lemas, etc.
\makeatletter
\def\th@plain{%
  \thm@notefont{}% same as heading font
  \itshape % body font
}
\def\th@definition{%
  \thm@notefont{}% same as heading font
  \normalfont % body font
}
\makeatother
\theoremstyle{plain}
\newtheorem{teo}{Teorema}
\newtheorem{lem}{Lema}
\newtheorem{prop}{Proposici\'on}
\newtheorem{cor}{Corolario}
\newtheorem{prob}{Problema Controlable}
\newtheorem{nota}{Notaci\'on}
\newtheorem{obs}{Observaci\'on}
\newtheorem{defi}{Definición}[section]
\setcounter{section}{1}
%%%%%%% inicio documento %%%%%%%
\begin{document}


\begin{framed}
\begin{enumerate}[\it a)]
		\item Tomando modulo a la primera igualdad tenemos:
\begin{eqnarray*}
|z_1+z_2| & = & |-u| \\
|z_1+z_2| & = & |u|	
\end{eqnarray*}	
Ocupando la desigualdad triangular tenemos:
$$
|u|=|z_1+z_2| \leq |z_1|+|z_2| = 2 
$$
Tomando modulo a la segunda igualdad tenemos:
\begin{eqnarray*}
	|z_1z_2| & = & |v| \\
	|z_1||z_2|  & = & |v| \\
	1 & = & |v|
\end{eqnarray*}
De donde concluimos.
	\item Dividiendo las igualdades del enunciado (podemos pues $|z_1z_2| \neq 0 $ y por ende distinto de 0) tenemos:
	\begin{eqnarray*}
	 \frac{z_1+z_2}{z_1z_2} & = & -\frac{u}{v} \\
  \frac{1}{z_1}+ \frac{1}{z_2} & = & -\frac{u}{v} \\ 
  \frac{\overline{z_1}}{|z_1|^2} +   \frac{\overline{z_2}}{|z_2|^2}  & = & -\frac{u}{v} \\
\overline{z_1}+\overline{z_2} & = &  -\frac{u}{v}\\
	\end{eqnarray*}
	
		\item Conjugando la primera igualdad del enunciado obtenemos $\overline{z_1+z_2 }= \overline{-u}$, combinando esto con $b)$ obtenemos:
		\begin{eqnarray*}
		\underbrace{\overline{z_1+z_2 }}_{\overline{-u}}	& = & \underbrace{\overline{z_1}+ \overline{z_2}}_{b)} \\
		-\overline{u }& = & -\frac{u}{v}  \\
	 \overline{u}v	& = & u
		\end{eqnarray*}
		\item En forma polar tenemos que $u=|u|e^{i\varphi}$ y que $v=\underbrace{|v|}_{=1}e^{i \theta}= e^{i \theta} $. Ocupando $c)$:
		\begin{eqnarray*}
		\overline{u}v	& = & u \\
		|u|e^{-i\varphi} e^{i \theta} & = & |u|e^{i\varphi} \\
	e^{i(\theta - \varphi)}	& = &  e^{i\varphi}
		\end{eqnarray*}
		De donde concluimos que existe $k$ tal que:
		$$
		\theta - \varphi = \varphi + 2k\pi 
		$$
		Y por tanto:
		$$
		\theta = 2\varphi + 2k \pi
		$$
		Que era lo buscado.
	\end{enumerate}
	\end{framed}
	\end{document}