\documentclass[letterpaper,11pt]{article}

\usepackage[activeacute,spanish]{babel}
\usepackage[left=1.8cm,top=1cm,right=1.8cm, bottom=1cm,letterpaper, includeheadfoot]{geometry}
\usepackage{framed}
\usepackage{babel}
\usepackage[utf8]{inputenc}
\usepackage{algorithmic}
\usepackage{algorithm}
%\usepackage{enumitem}
\usepackage{enumerate}
\usepackage{multicol}
\usepackage{amssymb, amsmath, amsthm}
\usepackage{subcaption}
\usepackage{graphicx,txfonts}
\usepackage{lmodern,url}
\usepackage{graphicx}
\usepackage{wrapfig}
\usepackage{hyperref}
\usepackage[dvipsnames]{xcolor}
\usepackage{epigraph}
\usepackage{color}
\usepackage{cancel}
\usepackage{tikz}
\def\checkmark{\tikz\fill[scale=0.4](0,.35) -- (.25,0) -- (1,.7) -- (.25,.15) -- cycle;} 
\floatname{algorithm}{Algoritmo}

\makeatletter


\setlength\epigraphwidth{8cm}
\setlength\epigraphrule{0pt}
\usepackage{fancyhdr}
\setlength{\headheight}{15pt} 
\pagestyle{fancy}
\fancypagestyle{plain}{%
    \fancyhf{}
    \lhead{\footnotesize\itshape\bfseries\rightmark}
    \rhead{\footnotesize\itshape\bfseries\leftmark}
    }

\setlength{\parindent}{1cm}
\newenvironment{chapquote}[2][2em]
  {\setlength{\@tempdima}{#1}%
   \def\chapquote@author{#2}%
   \parshape 1 \@tempdima \dimexpr\textwidth-2\@tempdima\relax%
   \itshape}
  {\par\normalfont\hfill--\ \chapquote@author\hspace*{\@tempdima}\par\bigskip}
\makeatother

% macros
\newcommand{\heart}{\ensuremath\heartsuit}
\newcommand{\grad}{\hspace{-2mm}$\phantom{a}^{\circ}$}
\newcommand{\Q}{\mathbb Q}
\newcommand{\R}{\mathbb R}
\newcommand{\N}{\mathbb N}
\newcommand{\Z}{\mathbb Z}
\newcommand{\C}{\mathbb C}
\newcommand{\U}{\mathcal U}
\newcommand{\ssi}{\Longleftrightarrow} %si y solo si
\newcommand{\To}{\Rightarrow}      %implica
\newcommand{\tq}{\mid }            % tal que
\newcommand{\exclusivo}{\veebar }  % o exclusivo
\renewcommand{\vec}[2]{\left(\begin{array}{c}{#1}\\{#2}\end{array}\right)}
\newcommand{\texii}[2]{\begin{minipage}{0.5\textwidth} #1 \end{minipage}  
                     \begin{minipage}{0.5\textwidth} #2 \end{minipage}}

%%%operadores matematicos
\providecommand{\abs}[1]{\lvert#1 \rvert}
\providecommand{\pin}[2]{\left< #1,#2 \right>} %producto interno
\providecommand{\dpartial}[2]{\frac{\partial #1}{\partial #2}} %derivada parcial


%Teoremas, Lemas, etc.
\theoremstyle{plain}
\newtheorem{teo}{Teorema}
\newtheorem{lem}{Lema}
\newtheorem{prop}{Proposici\'on}
\newtheorem{cor}{Corolario}
\newtheorem{prob}{Problema Controlable}
\newtheorem{nota}{Notaci\'on}
\newtheorem{obs}{Observaci\'on}

%%%%%%% inicio documento %%%%%%%
\begin{document}

%============Encabezado estandar============
\newpage
\pagestyle{fancy}
\fancyhf{}
\fancyhead[L]{\textit{Facultad de Ciencias Físicas y Matemáticas}}
\fancyhead[R]{\textit{Universidad de Chile}}

\begin{wrapfigure}{R}{0.2\textwidth} %this figure will be at the right
    \vspace{-5mm}
    \includegraphics[width=0.2\textwidth]{img/fcfm2.png}
\end{wrapfigure}


\noindent
\textbf{MA1101-1 Introducción al Álgebra}\\
\textbf{Profesor: }Leonardo Sánchez C.\\
\textbf{Auxiliar: }Marcelo Navarro

\begin{center}
{\bf \Large Control Recuperativo 2011} \\
{07 de Junio de 2018}
\end{center}

\begin{enumerate}[\bf P1.]

\item Demuestre por inducción que
    \begin{enumerate}
        \item $\displaystyle \forall n\in \N, n\geq 1, \frac{1}{n+1}+\frac{1}{n+2}+\dots+\frac{1}{2n+1}\leq \frac{5}{6}$
        \item Sea $k_0\in \N$ un número natural fijo cualquiera y $n\in \N$ un número impar cualquiera, demuestre sin uso de inducción que la suma de los $n$ naturales consecutivos a partir de $k_0$ es divisible por $n$.
    \end{enumerate}

\item Se define en $\Z \times \Z \setminus \{0\}$ la relación $\rho$ por $(x,y)\rho (z,t) \ssi xt=zy$
    \begin{enumerate}
        \item Demuestre que $\rho$ es de equivalencia y describa explícitamente $[(0,1)]$ y $[(3,3)]$.
        \item Sea $f: \Z \times \Z \setminus \{0\} \to \Q$ dada por
            $$f(x,y)=\frac{x}{y}$$
            Demuestre que $(x,y)\rho (z,t) \ssi f(x,y)=f(z,t)$
        \item Demuestre que la función $F: (\Z \times \Z \setminus \{0\})/_{\rho} \to \Q$ dada por
            $$F([(x,y)])=f(x,y)$$
            es biyectiva
    \end{enumerate}
\end{enumerate}

\begin{center}
    \LARGE \bf Aritmética Modular
\end{center}
\begin{enumerate}
\item[\bf P3.] ¿Es posible encontrar un entero que sea un cuadrado perfecto, que sea divisible por 2 pero no divisible por 4?
\item[\bf P4.] 
    Sea $n$ un natural mayor o igual que dos. Si $x$ es una solución entera no trivial de la ecuación \\$x^2=1 ~ (mod~ n)$, es decir $x\in \Z$, $x\neq 1 ~(mod~ n)$ y $x\neq -1~ (mod~ n)$.
    \begin{enumerate}
        \item Probar que $n$ divide a $(x-1)(x+1)$
        \item Si $n$ es un número primo, probar que no existen soluciones enteras no triviales de la ecuación.
        \item Probar que si $n=p\cdot q$, donde $p$ y $q$ son números enteros primos distintos, entonces el\\ $MCD(n,x+1) \in \{p,q\}$, donde $MCD$ es el máximo común divisor.
    \end{enumerate}
    
    Hint: \textit{teorema fundamental de la aritmética: todo número natural mayor que 1 es un primo o bien se puede representar por un único producto de primos (estos primos se pueden repetir)}
\end{enumerate}


\newpage 

\begin{center}
    Soluciones Aritmética Modular
\end{center}

\begin{enumerate}
\item[\bf P3.] Planteemos una ecuación que nos ayude a interpretar lo que nos piden.

En primer lugar nos piden encontrar un cuadrado perfecto, que podemos denotarlo por $x^2$, tal que sea divisible por 2 pero no divisible por 4. Esto ultimo lo podemos ver con ayuda de escribir la secuencia de números que lo cumplen. 

Notar que los números que cumplen la propiedad pedida son:

$$2,6,10,14,18,22,\dots$$

Una forma de escribirlos puede ser $2+4n, ~ n\in \N$, claramente este número es divisible por $2$ y no es divisible por $4$.

Ahora juntando todo, planteamos la siguiente ecuación:

$$ x^2=2+4n $$

Pero hay un gran problema, tenemos una ecuación y hay $1$ incógnita y una variable que puede recorrer todo $\N$. Lo que haremos será ``aplicar" modulo congruencia $4$ en la igualdad anterior.

por lo que obtenemos lo siguiente:
\begin{align*} 
 & x^2=2+4n  \\ 
\Longrightarrow  & x^2= 2+4n \ (mod ~ 4)\\
\Longleftrightarrow  & x^2\equiv_{4} 2+4n \\
\Longleftrightarrow  & x^2\equiv_{4} 2 \\
\end{align*}

Y esto ultimo es una ecuación modular.

Veamos como se resuelve este tipo de ecuación y si es que tiene solución.

Una ecuación del estilo $x\equiv_n a$ tiene como solución los elementos del conjunto $\{x\in \Z : x \equiv_n a\}$ ¿A qué se parece esto?. Bueno, es la clase de equivalencia de $a$ por lo que podemos decir que la solución es $[a]_{\equiv_n}$.

Volvamos al problema...

tenemos que ver si $x^2\equiv_{4} 2 $ tiene solución. En este caso, lo que haremos es \textbf{probar todos los enteros}, oye pero Marcelo, los enteros son infinitos... ¿como los probaremos todos? Tranquilo estimado estudiante, probaremos todos los enteros ya que gracias a las clases de equivalencia sabremos como se comportan todos los números en base a los representantes de su clase.

En modulo congruencia 4, tenemos las siguientes 4 clases de equivalencia

$[0]=\{x\in \Z : x \equiv_4 0\}=\{ x\in \Z  : x=4k, k\in \Z\}=\{0,-4,4,-8,8,-16,16,\dots \}$\\
$[1]=\{x\in \Z : x \equiv_4 1\}=\{ x\in \Z  : x=4k+1, k \in \Z\}$\\
$[2]=\{x\in \Z : x \equiv_4 2\}=\{ x\in \Z  : x=4k+2, k \in \Z\}$\\
$[3]=\{x\in \Z : x \equiv_4 3\}=\{ x\in \Z  : x=4k+3, k \in \Z\}$

Por lo que debemos probar 4 números en total para ir tanteando cual es la solución. Lo que haremos será del siguiente estilo. Si tienen una ecuación de la forma $3x=12$, vamos a ir probando si $x=1$ sirve, o bien $x=2$ sirve, así hasta encontrar la solución.

Bueno tenemos 4 casos para resolver la ecuación $x^2\equiv_{4} 2$:

\begin{itemize}
    \item \text{[Caso $x\in [0]$]}: Aquí podemos tomar el representante (pues todos los elementos harían lo mismo en la ecuación, ya que están en la misma clase). Si $x=0$, reemplazando en la ecuación nos queda que
    $$0^2\equiv_{4} 2  \iff 0 \equiv_4 2$$
    Lo cual es falso, ya que $0\in [0]$ y $2 \in [2]$ y las clases son distintas. Por lo que descartamos que haya alguna solución de la ecuación en este caso.
    
    \item \text{[Caso $x\in [1]$]}: Idéntico a lo anterior, si $x=1$, reemplazamos en la ecuación y obtenemos 
    $$ 1^2\equiv_{4} 2  \iff 1 \equiv_4 2$$
    Lo cual es falso, ya que $1\in [1]$ y $2 \in [2]$ y las clases son distintas. Se descarta solución.
    
    \item \text{[Caso $x\in [2]$]}: Idéntico a lo anterior, si $x=2$, reemplazamos en la ecuación y obtenemos 
    $$ 2^2\equiv_{4} 2  \iff 4 \equiv_4 2 \iff 0 \equiv_4 2$$
    Lo cual es falso, ya que es el mismo caso que en el primero punto. Se descartan soluciones en este caso.
    
    \item \text{[Caso $x\in [3]$]}: Idéntico a lo anterior, si $x=3$, reemplazamos en la ecuación y obtenemos 
    $$ 3^2\equiv_{4} 2  \iff 9 \equiv_4 2 \iff 8+1 \equiv_4 2 \iff 1 \equiv_4 2$$
    Lo cual es falso, ya que es lo mismo que el segundo punto. Se descarta encontrar soluciones en este caso.
\end{itemize}

Luego de probar todos los casos, ninguno nos sirvió para encontrar un valor $x$ que solucione la ecuación $x^2 \equiv_4 2$. Por lo que no hay solución y, en virtud de esto, no existe número entero que cumpla que sea un cuadrado perfecto, múltiplo de $2$ y no múltiplo de $4$, ya que si existiera seria solución de la ecuación planteada.


\item[\bf P4.] 
\begin{enumerate}
    \item En efecto, como $x^2=1 ~ (mod ~n)$ es lo mismo que $x^2 \equiv_n 1$, y por definición
    $$x^2 \equiv_n 1 \iff x^2-1=nk, ~ k \in \Z$$
    Esto ultimo indica que que $n$ es multiplo de $x^2-1$ o bien divide a $x^2-1$. Se concluye factorizando la diferencia de cuadrados.
    $$n ~|~ x^2-1 \iff n ~|~(x-1)(x+1)$$
     
    Donde $a ~|~ b$ significa que $a$ divide a $b$
     
    \item Por contradicción, sea $n$ primo y sea $x \in \Z$ una solución no trivial de la ecuación $x^2=1 ~ (mod ~n)$. 
    
    Por $a)$, sabemos que $$n ~|~ (x+1)(x-1)$$.
    
    Ahora bien, si $n$ divide al producto $(x+1)(x-1)$ pueden sucedes muchas cosas
    \begin{enumerate}
        \item $n$ divide a $(x+1)(x-1)$ pero no a $(x+1)$ o $(x-1)$ en solitario.
        
        Por ejemplo 6 divide a $3\cdot2$ pero no divide a $3$ ni a $2$.
        \item $n$ divide solamente a $(x+1)$.
        \item $n$ divide solamente a $(x-1)$.
        \item $n$ divide a $(x+1)$ y simultáneamente a $(x-1)$.
    \end{enumerate}
    
    El caso $1$ se descarta, ya que, por hipotesis, $n$ es primo por lo que no se puede generar por producto de dos números (como ocurre con el ejemplo dado). Luego los casos restantes son validos y pueden llegar a ocurrir, por lo que se puede concluir que 
    
    $$ n~ | ~ (x+1) ~~~\lor~~~ n~ | ~ (x-1) $$
    
    Esto abarca los casos validos (verificar).
    
    Luego, sin perdida de generalidad, supongamos que se cumple que $n~ | ~ (x-1)$, esto significa que $x-1=nk, k\in \Z$, o equivalentemente, $x\equiv_n 1$. Es decir, $x$ es solución trivial del problema, lo que es una contradicción ya que asumimos que $x$ es solución no trivial. Analize usted el caso de que $ n~ | ~ (x-1)$.
    
    Por lo tanto, Se cumple lo que habia que probar.
    
    \item En efecto, sea $n=pq$. Por $a)$ sabemos que $n~|~(x-1)(x+1)$. Luego por $b)$ sabemos que $n$ tiene que dividir a $(x-1)(x+1)$ pero no a $(x+1)$ o $(x-1)$ en solitario, ya que si esto ultimo ocurriera, $x$ seria solución trivial de la ecuación.
    
    Por lo tanto, $n~|~(x-1)(x+1)$ o bien $pq~|~(x-1)(x+1)$. Luego, debido a lo anterior, debe ocurrir que o bien $p$ o bien $q$ divida solamente a $(x-1)$ y el restante divida a $(x+1)$. En virtud del teorema fundamental de la aritmética, si $(x+1)=r_{1}r_{2}\dots r_{\ell}$ es su descomposición prima, y si $p$ divide a $(x+1)$, como $p$ es primo, $p$ debe aparecer en la descomposición prima de $(x+1)$ es decir $p=r_j$ para algún $j\in \{1, \dots, \ell\}$. Luego $p$ divide a $(x+1)$, donde en la descomposición prima de $(x+1)$ aparece $p$.
    
    Luego el $MCD(n,x+1)=p$ ya que $p$ divide a $n$ directamente y $x+1$ tiene como factor dentro de su descomposición prima a $p$.
    
    Finalmente, si intercambiamos roles entre $p$ y $q$ (es decir, si en vez de suponer que $p$ divide a $(x+1)$, suponemos que $q$ divide a $(x+1)$). Se concluye que $MCD(n,x+1)\in \{p,q\}$
\end{enumerate}
    
\end{enumerate}


\end{document}