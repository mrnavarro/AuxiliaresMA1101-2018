\documentclass[letterpaper,11pt]{article}

\usepackage[activeacute,spanish]{babel}
\usepackage[left=1.8cm,top=1cm,right=1.8cm, bottom=1cm,letterpaper, includeheadfoot]{geometry}
\usepackage{framed}
\usepackage{babel}
\usepackage[utf8]{inputenc}
\usepackage{algorithmic}
\usepackage{algorithm}
%\usepackage{enumitem}
\usepackage{enumerate}
\usepackage{multicol}
\usepackage{amssymb, amsmath, amsthm}
\usepackage{subcaption}
\usepackage{graphicx,txfonts}
\usepackage{lmodern,url}
\usepackage{graphicx}
\usepackage{wrapfig}
\usepackage{hyperref}
\usepackage[dvipsnames]{xcolor}
\usepackage{epigraph}
\usepackage{color}
\usepackage{cancel}
\usepackage{tikz}
\def\checkmark{\tikz\fill[scale=0.4](0,.35) -- (.25,0) -- (1,.7) -- (.25,.15) -- cycle;} 
\floatname{algorithm}{Algoritmo}

\makeatletter


\setlength\epigraphwidth{8cm}
\setlength\epigraphrule{0pt}
\usepackage{fancyhdr}
\setlength{\headheight}{15pt} 
\pagestyle{fancy}
\fancypagestyle{plain}{%
    \fancyhf{}
    \lhead{\footnotesize\itshape\bfseries\rightmark}
    \rhead{\footnotesize\itshape\bfseries\leftmark}
    }

\setlength{\parindent}{1cm}
\newenvironment{chapquote}[2][2em]
  {\setlength{\@tempdima}{#1}%
   \def\chapquote@author{#2}%
   \parshape 1 \@tempdima \dimexpr\textwidth-2\@tempdima\relax%
   \itshape}
  {\par\normalfont\hfill--\ \chapquote@author\hspace*{\@tempdima}\par\bigskip}
\makeatother

% macros
\newcommand{\heart}{\ensuremath\heartsuit}
\newcommand{\grad}{\hspace{-2mm}$\phantom{a}^{\circ}$}
\newcommand{\Q}{\mathbb Q}
\newcommand{\R}{\mathbb R}
\newcommand{\N}{\mathbb N}
\newcommand{\Z}{\mathbb Z}
\newcommand{\C}{\mathbb C}
\newcommand{\U}{\mathcal U}
\newcommand{\ssi}{\Longleftrightarrow} %si y solo si
\newcommand{\To}{\Rightarrow}      %implica
\newcommand{\tq}{\mid }            % tal que
\newcommand{\exclusivo}{\veebar }  % o exclusivo
\renewcommand{\vec}[2]{\left(\begin{array}{c}{#1}\\{#2}\end{array}\right)}
\newcommand{\texii}[2]{\begin{minipage}{0.5\textwidth} #1 \end{minipage}  
                     \begin{minipage}{0.5\textwidth} #2 \end{minipage}}

%%%operadores matematicos
\providecommand{\abs}[1]{\lvert#1 \rvert}
\providecommand{\pin}[2]{\left< #1,#2 \right>} %producto interno
\providecommand{\dpartial}[2]{\frac{\partial #1}{\partial #2}} %derivada parcial


%Teoremas, Lemas, etc.
\theoremstyle{plain}
\newtheorem{teo}{Teorema}
\newtheorem{lem}{Lema}
\newtheorem{prop}{Proposici\'on}
\newtheorem{cor}{Corolario}
\newtheorem{prob}{Problema Controlable}
\newtheorem{nota}{Notaci\'on}
\newtheorem{obs}{Observaci\'on}
\newcommand{\cupdot}{\mathbin{\mathaccent\cdot\cup}}
%%%%%%% inicio documento %%%%%%%
\begin{document}

%============Encabezado estandar============
\newpage
\pagestyle{fancy}
\fancyhf{}
\fancyhead[L]{\textit{Facultad de Ciencias Físicas y Matemáticas}}
\fancyhead[R]{\textit{Universidad de Chile}}

\begin{wrapfigure}{R}{0.2\textwidth} %this figure will be at the right
    \vspace{-5mm}
    \includegraphics[width=0.2\textwidth]{img/fcfm2.png}
\end{wrapfigure}


\noindent
\textbf{MA1101-1 Introducción al Álgebra}\\
\textbf{Profesor: }Leonardo Sánchez C.\\
\textbf{Auxiliar: }Marcelo Navarro

\begin{center}
{\bf \Large Auxiliar 12: Conjuntos Infinitos}\\
{\today}
\end{center}

\begin{framed}
	\begin{multicols}{2}
	    \begin{itemize}  
    \item $|A|=|B|$ si existe una función $f:A\to B$ biyectiva.
    
    \item $|A|\leq|B|$ si existe una función $f:A \to B$ inyectiva.
    
    \item $|A|<|B|$ si existe una función $f:A\to B$ inyectiva, pero no existe una función biyectiva $g:A \to B$
    
    \item Se tienen las siguientes propiedades:
        \begin{enumerate}
            \item $|A|\leq |A|$
            \item Si $A \subseteq B$, entonces $|A|\leq |B|$
            \item Si $|A|\leq |B|$ y $|B|\leq |C|$, entonces $|A|\leq |C|$
        \end{enumerate}
    
    \item \textbf{Teorema Cantor-Bernstein-Schöeder}
    $$|A|\leq |B| \land |B|\leq |A| \Rightarrow  |A|=|B|$$
    
    \item \textbf{Cardinal de la imagen de un conjunto}\\
    Si $f:A \to B$ es función, entonces $|f(A)|\leq |A|$
    
    \item $\N$ es infinito y si un conjunto $A$ cumple que $|A|=|\N|$, entonces se dirá que $A$ es numerable.
    Por otro lado si un conjunto $A$ cumple que $|A| \leq |\N|$ se dirá que $A$ es a lo más numerable.
    
    \item $|\N|$ es el menor cardinal infinito.
    
    \item Todo conjunto infinito $A$ inmediatamente cumple que $|A|\geq |\N|$. Es decir, $A$ es infinito si y sólo si $|A|\geq |\N|$.
    
    \item Sea $A$ un conjunto infinito tal que $|A|\leq |\N|$, entonces $A$ es numerable, es decir, $|A|=|\N|$
    
    \item Sea $A$ infinito y $B$ finito. Entonces $|A\cup B|=|A\setminus B|=|A|$
    
    \item $\Z$ y $\Q$ son numerables.
    
    \item Sea $A_1,A_2,\dots,A_n$ una colección finita de conjuntos numerables, entonces $\displaystyle \bigcup_{i=1}^{n}A_i$ también es numerable
    
    \item Sea $A_1,A_2,\dots,A_n$ una colección finita de conjuntos numerables, entonces $$\displaystyle \prod_{i=1}^{n}A_i= A_1 \times \cdots \times A_n$$ también es numerable.\\
        Una consecuencia de esto es que $\N^n , \Z^n$ y $\Q^n$ son numerables, con $n \in \N, ~ n\geq 1$
    
    \item Sea $(A_i)_{i\in \N}$ una colección numerable de conjuntos numerables, entonces $\displaystyle \bigcup_{i\in \N}A_i$ es numerable. 
    
    \item Sea $(A_i)_{i\in \mathcal{I}}, ~\mathcal{I} \subseteq \N$, una colección a lo más numerable de conjuntos a lo más numerables (i.e. $|A_i|\leq |N|$). Entonces
    $\displaystyle \bigcup_{i\in \mathcal{I}}A_i$ es a lo más numerable
    
    \item El producto de una familia numerable de conjuntos finitos de tamaño dos \emph{no es numerable}\\
    Ej: $\displaystyle \prod_{i \in \N} \{x_i,y_i\} ~ \text{ donde } x_i,y_i \in \R$ para todo $i \in \N$
    \item \textbf{Teorema Cantor}\\Sea $A$ un conjunto entonces $|A|<|\mathcal{P}(A)|$
    \item Un conjunto $A$ se dirá \emph{no numerable}  si $|\N|<|A|$
    \item $\R$, $\mathcal{P}(\N)$, $|(a,b)|$, $|(a,b]|$, $|[a,b)|$ y $|[a,b]|$    son \emph{no numerable}, con $a,b \in \R ~,a<b$ 
        \end{itemize}
    \end{multicols}
\end{framed}

\begin{enumerate}[\bf P1.]
    \item \textbf{[Strings y números binarios]}
        
        \begin{enumerate}
            \item Demuestre que el conjunto de todos los strings que se pueden generar a partir de símbolos en $\{0,1\}$, es infinito numerable.
            %definir sigma_estrella=union numerable de sigma finito, luego sigma_estrella es a lo más numerable (sigma_estrella <= N), luego inyectar de N a sigma_estrella mediante la representacion binaria de un numero. luego (N<=sigma_estrella), tambien se puede usar strings de la forma 101,1001,10001...->(10^n1)n en N este conjunto es subconjunto de sigma estrella y es infinito.
            \item Demuestre que el conjunto de todas las secuencias binarias infinitas es no numerable. %usar diagonalizacion de cantor
            \item \textbf{[Propuesto]} Demuestre que el conjunto de las novelas que pueden ser escritas en español es numerable.
        \end{enumerate}
    \textit{Nota: Un string o palabra es una cadena finita de caracteres}
    
    \item \textbf{[Infinito vs Finito]} 
    
    Demuestre que si $B\subseteq A$, $A$ es infinito y $B$ finito, entonces $A\setminus B$ es infinito. %contradiccion%
    
    \item \textbf{[Irracionales]}
    
    Demuestre que los irracionales son no numerables
    
    \item \textbf{[Verdadero o Falso]}
    
    Sean $A,B,C,D$ conjuntos. Demuestre o de un contraejemplo de:
        \begin{enumerate}
            \item Si $|A|=|B|$ y $|A\times C|=|B\times D|$, entonces $|C|=|D|$. %falsos ambos
            \item Si $|A|=|B|$, $A\cap C=\emptyset,~ B\cap D=\emptyset$ y $|A\cupdot C|=|B\cupdot D|$ entonces $|C|=|D|$.
            
            \textit{Donde el simbolo $\cupdot$ representa la union disjunta.}
            \item Si $A\subset B \Longrightarrow |A|<|B|$ 
            
            
        \end{enumerate}
    
    \item \textbf{[Triangulos]}
    
    Demuestre que el conjunto de todos los triangulos cuyos vertices son elementos de $\Q\times \Q$ es numerable.\\
    \textit{Propuesto: ¿Que ocurre si, en el mismo problema, en vez de pensar en triángulos pensamos en polígonos?}
    
    \item \textbf{[Conjunto de funciones]}
    
    Considere el conjunto $A\neq \emptyset$, se define el conjunto $\mathcal{F}=\{f:\{1,2,3\} \to A ~|~ f \text{ es función} \}$.
        \begin{enumerate}
            \item Demuestre que $|\mathcal{F}|=|A^{3}|$\\
            \textbf{Hint:} para $f\in \mathcal{F}$ considere la tupla $(f(1),f(2),f(3))$
            \item Demuestre que si $A$ es numerable, entonces $\mathcal{F}$ tambien lo es.
        \end{enumerate}
        
    \item \textbf{[Recorrido de un insecto]}
    
    Un insecto debe cubrir, saltando de izquierda a derecha, la distancia desde 0 a 1 en una recta. En cada punto de su recorrido, el insecto puede elegir entre saltar directamente hacia el uno (y así completar su viaje), o avanzar la mitad del tramo que le falta por cubrir.\\
    Pruebe que la colección de recorridos (secuencias de pasos) por los que puede optar nuestro insecto, es numerable.
    
    \item \textbf{[Suma de conjuntos]}
    
    Sean $A,B \subseteq \R$ numerables. Demuestre que el conjunto $A+B=\{a+b : a \in A, b \in B \}$ es numerable.
        
\end{enumerate}

\end{document}