\documentclass[letterpaper,10pt]{article}

\usepackage[activeacute,spanish]{babel}
\usepackage[left=1.8cm,top=1cm,right=1.8cm, bottom=1cm,letterpaper, includeheadfoot]{geometry}
\usepackage{framed}
\usepackage{babel}
\usepackage[utf8]{inputenc}
\usepackage{algorithmic}
\usepackage{algorithm}
%\usepackage{enumitem}
\usepackage{enumerate}
\usepackage{multicol}
\usepackage{amssymb, amsmath, amsthm}
\usepackage{subcaption}
\usepackage{graphicx,txfonts}
\usepackage{lmodern,url}
\usepackage{graphicx}
\usepackage{wrapfig}
\usepackage{hyperref}
\usepackage[dvipsnames]{xcolor}
\usepackage{epigraph}
\usepackage{color}
\usepackage{cancel}
\usepackage{tikz}
\def\checkmark{\tikz\fill[scale=0.4](0,.35) -- (.25,0) -- (1,.7) -- (.25,.15) -- cycle;} 
\floatname{algorithm}{Algoritmo}

\makeatletter


\setlength\epigraphwidth{8cm}
\setlength\epigraphrule{0pt}
\usepackage{fancyhdr}
\setlength{\headheight}{15pt} 
\pagestyle{fancy}
\fancypagestyle{plain}{%
    \fancyhf{}
    \lhead{\footnotesize\itshape\bfseries\rightmark}
    \rhead{\footnotesize\itshape\bfseries\leftmark}
    }

\setlength{\parindent}{1cm}
\newenvironment{chapquote}[2][2em]
  {\setlength{\@tempdima}{#1}%
   \def\chapquote@author{#2}%
   \parshape 1 \@tempdima \dimexpr\textwidth-2\@tempdima\relax%
   \itshape}
  {\par\normalfont\hfill--\ \chapquote@author\hspace*{\@tempdima}\par\bigskip}
\makeatother

% macros
\newcommand{\heart}{\ensuremath\heartsuit}
\newcommand{\grad}{\hspace{-2mm}$\phantom{a}^{\circ}$}
\newcommand{\Q}{\mathbb Q}
\newcommand{\R}{\mathbb R}
\newcommand{\N}{\mathbb N}
\newcommand{\Z}{\mathbb Z}
\newcommand{\K}{\mathbb K}
\newcommand{\C}{\mathbb C}
\newcommand{\U}{\mathcal U}
\newcommand{\ssi}{\Longleftrightarrow} %si y solo si
\newcommand{\To}{\Rightarrow}      %implica
\newcommand{\tq}{\mid }            % tal que
\newcommand{\exclusivo}{\veebar }  % o exclusivo
\renewcommand{\vec}[2]{\left(\begin{array}{c}{#1}\\{#2}\end{array}\right)}
\newcommand{\texii}[2]{\begin{minipage}{0.5\textwidth} #1 \end{minipage}  
                     \begin{minipage}{0.5\textwidth} #2 \end{minipage}}

%%%operadores matematicos
\providecommand{\abs}[1]{\lvert#1 \rvert}
\providecommand{\pin}[2]{\left< #1,#2 \right>} %producto interno
\providecommand{\dpartial}[2]{\frac{\partial #1}{\partial #2}} %derivada parcial
\DeclareMathOperator{\tr}{Tr}
\DeclareMathOperator{\Ker}{Ker}
\DeclareMathOperator{\Ima}{Im}
\DeclareMathOperator{\rango}{r}

%Teoremas, Lemas, etc.
\theoremstyle{plain}
\newtheorem{teo}{Teorema}
\newtheorem{lem}{Lema}
\newtheorem{prop}{Proposici\'on}
\newtheorem{cor}{Corolario}
\newtheorem{prob}{Problema Controlable}
\newtheorem{nota}{Notaci\'on}
\newtheorem{obs}{Observaci\'on}
\newcommand{\cupdot}{\mathbin{\mathaccent\cdot\cup}}
%%%%%%% inicio documento %%%%%%%
\begin{document}

%============Encabezado estandar============
\newpage
\pagestyle{fancy}
\fancyhf{}
\fancyhead[L]{\textit{Facultad de Ciencias Físicas y Matemáticas}}
\fancyhead[R]{\textit{Universidad de Chile}}

\begin{wrapfigure}{R}{0.2\textwidth} %this figure will be at the right
    \vspace{-5mm}
    \includegraphics[width=0.2\textwidth]{img/fcfm2.png}
\end{wrapfigure}


\noindent
\textbf{MA1101-1 Introducción al Álgebra}\\
\textbf{Profesor: }Leonardo Sánchez C.\\
\textbf{Auxiliar: }Marcelo Navarro

\begin{center}
{\bf \Large Auxiliar 14: Grupo y Anillos}\\
{\today}
\end{center}

\begin{enumerate}[\bf P1.]
    \item \textbf{[Calentamiento]}\\Sea $f : G \to G$, definida por $f(x)=x^{-1}$. Demuestre que:
	$$
	\text{$f$ es endomorfismo $ \iff $ $(G,*)$ es abeliano}
	$$
	En caso de que $(G,*)$ sea abeliano ¿Es $f$ un isomorfismo?.
    \item \textbf{[Varios de Lagrange]}
        \begin{enumerate}
            \item Sea $(G,*)$ un grupo tal que $|G|$ es finito.
                \begin{enumerate}
                    \item  Demuestre que si existe $x\in G\setminus \{e\}$ tal que $x=x^{-1}$, entonces $|G|$ es par. 
                	\item   Suponga ahora que $|G|=6$. Demuestre que no existe $x\in G$, tal que $x^4=e$ y $x\neq e$, $x^2 \neq e$.
                	\item   Suponga ahora que $|G|=4$. Encuentre el máximo número de subgrupos para $(G,*)$. 
                	
                	¿Es cierto que  $(G,*)\cong(\Z_4,+_4)$?
                \end{enumerate}
            \item Sea $(G,*)$ es un grupo con neutro $e$.
                \begin{enumerate}
                    \item Demuestre que si $H,K$ son subgrupos de $G$ entonces $H\cap K$ también lo es.
                    \item Demuestre que si $H,K$ son subgrupos de $G$, tales que $|H|=38$ y $|K|=55$. Demostrar que $H\cap K=\{e\}$.
                \end{enumerate}
        \end{enumerate}
    
    \item \textbf{[El núcleo y la imagen]}\\Sean $(G,*)$, $(H\cdot)$ grupos con neutro $e_G$ y $e_H$ respectivamente. Sea también $\varphi : G \to H$ un morfismo.
        \begin{enumerate}
        	\item Demuestre que $\varphi (G)$ es un subgrupo de $H$.
        	\item Definimos el \emph{kernel} de $\varphi$ como:
        	$$
        	\Ker (\varphi) = \{g \in G : \varphi(g)=e_H \}
        	$$
        	Demuestre que $\Ker (\varphi)$ es un subgrupo de $G$.
        	\item Demuestre que $\varphi$ es inyectiva si y sólo si $\Ker (\varphi) = \{ e_G \}$.
        	\item Suponga que $G$ y $H$ son finitos y que $|G|$ es par y $|H|$ impar. Demuestre que no existe un morfismo inyectivo de $G$ en $H$.
        \end{enumerate}

    \item \textbf{[Producto de Cuerpos]}
    
    Sea $(\K, + , \cdot)$ un cuerpo. Definimos las siguientes operaciones sobre $\K \times \K$:
        $$
        (a,b)\oplus (c,d)=(a+c,b+d) \hspace{10 mm} (a,b) \odot (c,d) = (a\cdot c, b\cdot d )
        $$
        Se sabe (no lo demuestre) que $(\K \times \K, \oplus , \odot )$ es un anillo conmutativo con unidad.
        \begin{enumerate}
        	\item Encuentre el neutro para $\oplus$ y el neutro para $\odot$.
        	\item Demuestre que $\forall (a,b) \in( \K \times \K )\setminus \{0_{\K\times \K}\}$:
        	$$
        	(a,b) \text{ es divisor del 0} \iff (a,b) \text{ no es invertible}  
        	$$
        	\item ¿Es $(\K \times \K, \oplus , \odot )$ un cuerpo? Argumente.
        \end{enumerate}

    \item \textbf{[Morfismos de Anillos]} \\
    Sean $(A,+,\cdot)$ y $(B,\oplus, \odot)$ dos anillos con unidad. Sea $f: A \to B$ un morfismo de anillos, es decir:
    $$	f(x+ y)=f(x)\oplus f(y) \hspace{10 mm} f(x \cdot y)=f(x ) \odot f(y) \hspace{10 mm} \hspace{10 mm} f(1_A)=1_B $$
        \begin{enumerate} 
            \item Demuestre que para todo $a \in A$, $f(a^{-1})=f(a)^{-1}$.  
            \item Demuestre que $f(0_A)=0_B$.
            \item Demuestre que si todo $a \in A$ es invertible salvo $0_A$, entonces $f$ es inyectiva.
        \end{enumerate}
        \newpage 
    \item \textbf{[Propuesto: Divisores del Cero]}\\
    Sea $(A,+,\cdot)$ un anillo conmutativo.
        \begin{enumerate}
        	\item Si $a\in A$ es un divisor del cero y $b\in A$ cualquiera, demuestre que si $a\cdot b\neq 0$, entonces $a\cdot b$ es un divisor del cero.
        	\item Demuestre que si el producto de dos elementos de $A$ es un divisor de cero, entonces al menos uno de ellos es un divisor de cero. 
        \end{enumerate}
    
    \item \textbf{[Propuesto: Más Anillos]}\\En $\Z^2$ se definen las siguientes leyes de composición interna: 
        \begin{align*}
        (a,b)\oplus (c,d)&=(a+c,b+d)\\
        (a,b)\odot (c,d)&=(ac,0)
        \end{align*}
        Sabiendo que $(\Z^2,\oplus)$ es grupo Abeliano (no lo demuestre).
        \begin{enumerate}
        \item[(i)] Verifique que $(\Z^2,\oplus,\odot)$ es un anillo conmutativo.
        \item[(ii)] Averigüe si $(\Z^2,\oplus,\odot)$ tiene unidad y/o divisores de cero. ¿Es $(\Z^2,\oplus,\odot)$ cuerpo? Justifique.
        \end{enumerate}
\end{enumerate}

\end{document}