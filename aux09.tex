\documentclass[letterpaper,11pt]{article}

\usepackage[activeacute,spanish]{babel}
\usepackage[left=1.8cm,top=1cm,right=1.8cm, bottom=1cm,letterpaper, includeheadfoot]{geometry}
\usepackage{framed}
\usepackage{babel}
\usepackage[utf8]{inputenc}
\usepackage{algorithmic}
\usepackage{algorithm}
%\usepackage{enumitem}
\usepackage{enumerate}
\usepackage{multicol}
\usepackage{amssymb, amsmath, amsthm}
\usepackage{subcaption}
\usepackage{graphicx,txfonts}
\usepackage{lmodern,url}
\usepackage{graphicx}
\usepackage{wrapfig}
\usepackage{hyperref}
\usepackage[dvipsnames]{xcolor}
\usepackage{epigraph}
\usepackage{color}
\usepackage{cancel}
\usepackage{tikz}
\def\checkmark{\tikz\fill[scale=0.4](0,.35) -- (.25,0) -- (1,.7) -- (.25,.15) -- cycle;} 
\floatname{algorithm}{Algoritmo}

\makeatletter


\setlength\epigraphwidth{8cm}
\setlength\epigraphrule{0pt}
\usepackage{fancyhdr}
\setlength{\headheight}{15pt} 
\pagestyle{fancy}
\fancypagestyle{plain}{%
    \fancyhf{}
    \lhead{\footnotesize\itshape\bfseries\rightmark}
    \rhead{\footnotesize\itshape\bfseries\leftmark}
    }

\setlength{\parindent}{1cm}
\newenvironment{chapquote}[2][2em]
  {\setlength{\@tempdima}{#1}%
   \def\chapquote@author{#2}%
   \parshape 1 \@tempdima \dimexpr\textwidth-2\@tempdima\relax%
   \itshape}
  {\par\normalfont\hfill--\ \chapquote@author\hspace*{\@tempdima}\par\bigskip}
\makeatother

% macros
\newcommand{\heart}{\ensuremath\heartsuit}
\newcommand{\grad}{\hspace{-2mm}$\phantom{a}^{\circ}$}
\newcommand{\Q}{\mathbb Q}
\newcommand{\R}{\mathbb R}
\newcommand{\N}{\mathbb N}
\newcommand{\Z}{\mathbb Z}
\newcommand{\C}{\mathbb C}
\newcommand{\U}{\mathcal U}
\newcommand{\ssi}{\Longleftrightarrow} %si y solo si
\newcommand{\To}{\Rightarrow}      %implica
\newcommand{\tq}{\mid }            % tal que
\newcommand{\exclusivo}{\veebar }  % o exclusivo
\renewcommand{\vec}[2]{\left(\begin{array}{c}{#1}\\{#2}\end{array}\right)}
\newcommand{\texii}[2]{\begin{minipage}{0.5\textwidth} #1 \end{minipage}  
                     \begin{minipage}{0.5\textwidth} #2 \end{minipage}}

%%%operadores matematicos
\providecommand{\abs}[1]{\lvert#1 \rvert}
\providecommand{\pin}[2]{\left< #1,#2 \right>} %producto interno
\providecommand{\dpartial}[2]{\frac{\partial #1}{\partial #2}} %derivada parcial


%Teoremas, Lemas, etc.
\theoremstyle{plain}
\newtheorem{teo}{Teorema}
\newtheorem{lem}{Lema}
\newtheorem{prop}{Proposici\'on}
\newtheorem{cor}{Corolario}
\newtheorem{prob}{Problema Controlable}
\newtheorem{nota}{Notaci\'on}
\newtheorem{obs}{Observaci\'on}

%%%%%%% inicio documento %%%%%%%
\begin{document}

%============Encabezado estandar============
\newpage
\pagestyle{fancy}
\fancyhf{}
\fancyhead[L]{\textit{Facultad de Ciencias Físicas y Matemáticas}}
\fancyhead[R]{\textit{Universidad de Chile}}

\begin{wrapfigure}{R}{0.2\textwidth} %this figure will be at the right
    \vspace{-5mm}
    \includegraphics[width=0.2\textwidth]{img/fcfm2.png}
\end{wrapfigure}


\noindent
\textbf{MA1101-1 Introducción al Álgebra}\\
\textbf{Profesor: }Leonardo Sánchez C.\\
\textbf{Auxiliar: }Marcelo Navarro

\begin{center}
{\bf \Large Auxiliar 9: Sumatorias II}\\
{28 de Junio de 2018}
\end{center}

\begin{framed}
	\begin{multicols}{2}
	    \begin{itemize}  
            \item Se define una doble sumatoria como una suma $\displaystyle \sum_{k=n}^{m} b_{k}$. En donde $\displaystyle b_k=\sum_{i=p}^{q}a_{i,k}$. Es posible escribir esto como $\displaystyle \sum_{k=n}^{m} b_{k}=\sum_{k=n}^{m}\sum_{i=p}^{q}a_{i,k}$\\
            Las sumatorias múltiples se definen del mismo modo.
    
            \item En una sumatoria múltiples, si los limites superiores e inferiores no dependen de los índices entonces se pueden intercambiar las sumatorias, es decir
            $$\displaystyle \sum_{k=1}^{n}\sum_{j=1}^{m}a_{k,j}=\sum_{j=1}^{m}\sum_{k=1}^{n}a_{k,j}$$
    
            \item $$\displaystyle \sum_{k=1}^{n}\sum_{j=1}^{m}(a_{k}\cdot b_{j})= \sum_{k=1}^{n}(a_{k}\cdot \sum_{j=1}^{m} b_{j}) $$
            Esto ocurre siempre que $a_{k}$ no dependa del índice $j$, pues se considera constante para la sumatoria más interna
            
            \item Factorial de número:
                $$n!=n\cdot (n-1)! , \text{ donde } 0!=1$$
                $$n!=n\cdot(n-1)\cdot \dots \cdot 2\cdot 1  $$
        \end{itemize}
    \end{multicols}
\end{framed}

\begin{enumerate}[\bf P1.]
    \item \textbf{[¿Como escribo una sumatoria? :'c]}
    
        Considere, para $n\in \N \setminus \{0\}$ la suma
    $$ S=1+\dfrac{1+2}{2}+\dfrac{1+2+3}{3}+\cdots+\dfrac{1+2+3+\dots+n}{n} $$
    Escriba S como una sumatoria doble y calcule su valor.
    
    \item \textbf{[Más sumatorias]}
        \begin{enumerate}
            \item Sea $m\in \N, m\geq 5$. Calcule:
            $$\displaystyle \sum_{i=5}^{m} \sum_{j=1}^{i}\dfrac{i+1}{j(j+1)}$$
            
            \item Sea $n \in \N$ y $\Omega = \{(n-p)\in \N: 0\leq p< n ~,p\in \N \}$. Calcule:
            $$\displaystyle \sum_{j\in \Omega} \sum_{k=1}^{j}(k+\dfrac{2^j}{j})$$
        \end{enumerate}
        
    \item \textbf{[Intercambio de sumatorias]}
        \begin{enumerate}
            \item Demostrar que 
                $$\sum_{i=1}^n \sum_{j=1}^i a_{i,j}=\sum_{j=1}^n \sum_{i=j}^n a_{i,j}$$
            \item Demuestre que $\forall n \in \N \cup \{0\}$ y $a_1,a_2,a_3,\ldots\in \R$:
                $$
                \sum_{k=1}^n \sum_{i=1}^k a_i =\sum_{i=1}^n (n-i+1)a_i 
                $$
            \item Demuestre que si $a_{n,m}$ es una secuencia, entonces:
                $$
                \displaystyle \sum_{i=1}^n \sum_{\substack{j: j=0 \mod i \\ 0<j\leq n}} a_{i,j} = \sum_{j=1}^n \sum_{i: i \text{ divide a }j}a_{i,j}
                $$

        \end{enumerate}
    \item \textbf{[Suma a través de conjunto de índices disjuntos]} 
    \begin{enumerate}
        \item Se pide calcular en función de $n$, el valor de la suma
        $$\displaystyle \sum_{k=1}^{2n}(-1)^{k}k^{2} $$
        procediendo como se indica:
        \begin{enumerate}
            \item Escriba la suma de los términos pares usando $k=2i$, con $i \in \{1,\dots,n \}$.
            \item Escriba la suma de los términos impares usando $k=2i-1$, con $i \in \{1,\dots,n \}$.
        \end{enumerate}
        Use esto para calcular la suma pedida al inicio.
        \item \textbf{[Propuesto]} Sea $n\in \N$ positivo. Calcule
        $$\displaystyle \sum_{k=1}^{2n} \frac{1}{(3+(-1)^k)^{k}}  $$
    \end{enumerate}
    
\end{enumerate}

\end{document}