\documentclass[letterpaper,11pt]{article}

\usepackage[activeacute,spanish]{babel}
\usepackage[left=1.8cm,top=1cm,right=1.8cm, bottom=1cm,letterpaper, includeheadfoot]{geometry}
\usepackage{framed}
\usepackage{babel}
\usepackage[utf8]{inputenc}
\usepackage{algorithmic}
\usepackage{algorithm}
%\usepackage{enumitem}
\usepackage{enumerate}
\usepackage{multicol}
\usepackage{amssymb, amsmath, amsthm}
\usepackage{subcaption}
\usepackage{graphicx}
\usepackage{lmodern,url}
\usepackage{graphicx}
\usepackage{wrapfig}
\usepackage{hyperref}
\usepackage[dvipsnames]{xcolor}
\usepackage{epigraph}
\usepackage{color}
\usepackage{cancel}
\usepackage{tikz}
\def\checkmark{\tikz\fill[scale=0.4](0,.35) -- (.25,0) -- (1,.7) -- (.25,.15) -- cycle;} 
\floatname{algorithm}{Algoritmo}

\makeatletter


\setlength\epigraphwidth{8cm}
\setlength\epigraphrule{0pt}
\usepackage{fancyhdr}
\setlength{\headheight}{15pt} 
\pagestyle{fancy}
\fancypagestyle{plain}{%
    \fancyhf{}
    \lhead{\footnotesize\itshape\bfseries\rightmark}
    \rhead{\footnotesize\itshape\bfseries\leftmark}
    }

\setlength{\parindent}{1cm}
\newenvironment{chapquote}[2][2em]
  {\setlength{\@tempdima}{#1}%
   \def\chapquote@author{#2}%
   \parshape 1 \@tempdima \dimexpr\textwidth-2\@tempdima\relax%
   \itshape}
  {\par\normalfont\hfill--\ \chapquote@author\hspace*{\@tempdima}\par\bigskip}
\makeatother

% macros
\newcommand{\grad}{\hspace{-2mm}$\phantom{a}^{\circ}$}
\newcommand{\Q}{\mathbb Q}
\newcommand{\R}{\mathbb R}
\newcommand{\N}{\mathbb N}
\newcommand{\Z}{\mathbb Z}
\newcommand{\C}{\mathbb C}
\newcommand{\U}{\mathcal U}
\newcommand{\ssi}{\Longleftrightarrow} %si y solo si
\newcommand{\To}{\Rightarrow}      %implica
\newcommand{\tq}{\mid }            % tal que
\newcommand{\exclusivo}{\veebar }  % o exclusivo
\renewcommand{\vec}[2]{\left(\begin{array}{c}{#1}\\{#2}\end{array}\right)}
\newcommand{\texii}[2]{\begin{minipage}{0.5\textwidth} #1 \end{minipage}  
                     \begin{minipage}{0.5\textwidth} #2 \end{minipage}}

%%%operadores matematicos
\providecommand{\abs}[1]{\lvert#1 \rvert}
\providecommand{\pin}[2]{\left< #1,#2 \right>} %producto interno
\providecommand{\dpartial}[2]{\frac{\partial #1}{\partial #2}} %derivada parcial


%Teoremas, Lemas, etc.
\theoremstyle{plain}
\newtheorem{teo}{Teorema}
\newtheorem{lem}{Lema}
\newtheorem{prop}{Proposici\'on}
\newtheorem{cor}{Corolario}
\newtheorem{prob}{Problema Controlable}
\newtheorem{nota}{Notaci\'on}
\newtheorem{obs}{Observaci\'on}

%%%%%%% inicio documento %%%%%%%
\begin{document}

%============Encabezado estandar============
\newpage
\pagestyle{fancy}
\fancyhf{}
\fancyhead[L]{\textit{Facultad de Ciencias Físicas y Matemáticas}}
\fancyhead[R]{\textit{Universidad de Chile}}

\begin{wrapfigure}{R}{0.2\textwidth} %this figure will be at the right
    \vspace{-5mm}
    \includegraphics[width=0.2\textwidth]{img/fcfm2.png}
\end{wrapfigure}


\noindent
\textbf{MA1101-1 Introducción al Álgebra}\\
\textbf{Profesor: }Leonardo Sánchez C.\\
\textbf{Auxiliar: }Marcelo Navarro

\begin{center}
{\bf \Large Auxiliar 1: Lógica y Cuantificadores}\\
{28 de Marzo de 2018}
\end{center}

\begin{framed}
		\begin{multicols}{2}
			\begin{itemize}
			    \item Una proposición lógica es un enunciado que toma un valor de verdad $V$ o $F$.
				\item Los conectivos lógicos son operaciones entre proposiciones y permiten construir nuevas proposiciones a partir de proposiciones ya conocidas.
				\item Las tablas de verdad de las proposiciones básicas son:
				$$
				\begin{array}{|c||c|}
				\hline p &  \overline p \\ 
				\hline V & F \\ 
				\hline F & V \\ 
				\hline 
				\end{array} \hspace{5 mm}\begin{array}{|c|c||c|}
				\hline p & q & p \vee q \\ 
				\hline V & V & V \\ 
				\hline V & F & V \\ 
				\hline F & V & V \\ 
				\hline F & F & F \\ 
				\hline 
				\end{array} \hspace{5 mm}\begin{array}{|c|c||c|}
				\hline p & q & p \wedge q \\ 
				\hline V & V & V \\ 
				\hline V & F & F \\ 
				\hline F & V & F \\ 
				\hline F & F & F \\ 
				\hline 
				\end{array} 
				$$
				\item Los conectivos $\vee$ y $\wedge$ son asociativos, conmutativos y distribuyen uno con respecto a otro. 
				\item Otras proposiciones conocidas son:
				\begin{enumerate}
					\item $(p \implies q ) \equiv (\overline{p} \vee q)$
					\item $(p \iff q) \equiv [(p \implies q ) \wedge (q \implies p)]$
					\item $(p \veebar q ) \equiv \overline{(p \iff q)}$
				\end{enumerate}
				\item Se dirá que una proposición es una tautología si su valor de verdad es siempre $V$, en cambio si es siempre $F$ diremos que es una contradicción.
				\item Algunas tautologías útiles son:
				\begin{enumerate}
					\item $\overline{(p \vee q)} \iff (\overline{p} \wedge \overline{q})$
					\item $\overline{(p \wedge q)} \iff (\overline{p} \vee \overline{q})$
					\item $(p \implies q) \iff (\overline{q} \implies \overline{p})$
					\item $[(p \implies q)\wedge(q \implies r)]\implies (p \implies r)$
				\end{enumerate}
				\item Una función proposicional es una expresión $p(x)$, tal que al reemplazar $x$ en la función esta se transforma en una proposición $p(x)$.
			
			\item Un cuantificador nos proporciona información sobre los objetos a evaluar en la función proposicional. Los clásicos cuantificadores son :
			\begin{enumerate}
				\item Cuantificador Universal $(\forall)$, se lee ``para todo''.
				\item Cuantificador Existencial $(\exists)$, se lee ``existe''.
				\item Cuantificador de Existencia y Unicidad $(\exists !)$, se lee ``existe un único''.
			\end{enumerate}
			\item Las negaciones clásicas con cuantificadores son:
			\begin{enumerate}
				\item $\overline{[\forall x , p (x)]} \iff [\exists x, \overline{p(x)}]$
				\item $\overline{[\exists x, p(x)]}\iff [\forall x , \overline{p(x)}]$
				\item $\overline{[\exists! x, p(x)]}\iff$ \\ $[\forall x , \overline{p(x)}] \lor [\exists x,y , p(x)\land p(y)\land x\neq y]$
			\end{enumerate}
			
			\item La proposición $x \in A$ se lee $x$ pertenece al conjunto $A$.
			\end{itemize}	
		\end{multicols}
\end{framed}

\begin{enumerate}[\bf P1.]
    \item \textbf{[Nuevo operador]}\\
    Sean $p$ y $q$ proposiciones. Se define la proposición
    $p *q$, por la siguiente
    tabla de verdad:
        \begin{center}
            \begin{tabular}{|c|c|c|}
                \hline $p$ & $q$ & $p  *q$\\
                \hline V & V & F\\
                \hline V & F & F\\
                \hline F & V & F\\
                \hline F & F & V\\
                \hline
            \end{tabular}
        \end{center}
        \begin{enumerate}
            \item Probar que $\sim p \ \Leftrightarrow \ (p * p)$ y que $(p \lor q) \Leftrightarrow \ \sim(p * q)$.
            \item Expresar las proposiciones $(p\Rightarrow q)$ y $(q \land p)$ usando sólo $*$ y $\sim$.
        \end{enumerate}

    %\item \textbf{[Descubriendo valores]}\\
    %Determine los valores de verdad de las proposiciones $p, q, r, s$ y $t$, si se sabe que la proposición:
     %   \begin{center}
      %      $[~(p \ssi q) \land \overline{(r \Rightarrow s)}\land \overline{t}~] \Longrightarrow [s \lor (q \Rightarrow s)]$ es falsa
       % \end{center}
    \item \textbf{[Métodos de Demostración]} \\
		Sean $p,q,r,s$ proposiciones lógicas.
		\begin{enumerate}
			\item Demuestre mediante el método algebraico o simbólico que:
			\begin{enumerate}
				\item $p \lor q \lor (\overline{p} \land \overline {q} )   $.
				\item $(\overline{p} \land \overline{q}) \iff \overline{(p \lor (\overline{p}  \land q ))}$.
				\item $[(p \implies \overline{q}) \land (r \implies q)] \implies (p \implies \overline{r})$.
			\end{enumerate}
			\item Demuestre mediante el método exploratorio que:
			\begin{enumerate}
				\item $
				(p \implies r) \implies ((p\wedge q)\implies r)
				$.
				\item $[(\overline{p} \lor q) \lor (\overline{r} \land \overline{p})]\Longleftrightarrow (p\Rightarrow q)$
			\end{enumerate}
			\item Demuestre por contradicción:
			\begin{enumerate}
				\item $[(p \Rightarrow q) \land (\overline{s} \Rightarrow \overline{r})] \Longrightarrow [\overline{p} \lor \overline{r} \lor (q \land s)]$
			\end{enumerate}
		\end{enumerate}
		
    %\item Demuestre, sin usar tablas de verdad, que
	%		$$[(p \lor q) \iff r] \implies [(q \implies r) \land (p \implies r)]$$
	
    \item \textbf{[Cuantificadores]} \\
    Sea $A=\{1,\ 2,\ 3,\ 4,\ 5,\ 6  \}$. Escribir en símbolos matemáticos, escribir su negación y averiguar el valor de verdad de las siguientes proposiciones:
        \begin{enumerate}
            \item Hay un elemento en A que es mayor que los restantes.
            \item Existe un único elemento en A  cuyo cuadrado es 4.
            \item Para cada elemento en A existe otro en A que es menor o igual que él.
            \item Existe un elemento cuyo cuadrado es igual a sí mismo.
        \end{enumerate} 
        
    \item \textbf{[¿Todas las proposiciones?]} \\
    Demuestre que $[\exists y] [p(y)\Rightarrow (\forall{x})p(x)]$, donde $p(\cdot)$ es una función proposicional.
    
    \item \textbf{[Si es que alcanzamos... Aplicación de Lógica]} \\Si los 123 residentes de un edificio tienen edades que suman 3813 años, entonces existen 100 de ellos cuyas edades suman al menos 3100 años.

    %\item ¿Es $[(p\Rightarrow q) \Rightarrow q] \Rightarrow q$ una tautología?
    %\item Se define recursivamente $\sigma_{k}$ como sigue: $\sigma_{0}=p\Rightarrow q$ y $\sigma_{k+1}=\sigma_{k}\Rightarrow q$. ¿Para que valores de $k$, $\sigma_{k}$ es verdadero?
    
    \item \textbf{[Para la casa]}
        \begin{enumerate}
            \item ¿Es $[(p\Rightarrow q) \Rightarrow q] \Rightarrow q$ una tautología?
            \item Se define recursivamente $\sigma_{k}$ como sigue: $\sigma_{0}=p\Rightarrow q$ y $\sigma_{k+1}=\sigma_{k}\Rightarrow q$. ¿Para que valores de $k$, $\sigma_{k}$ es verdadero?
        \end{enumerate}
        
\end{enumerate}


\end{document}
