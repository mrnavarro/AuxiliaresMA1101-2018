\documentclass[letterpaper,11pt]{article}

\usepackage[activeacute,spanish]{babel}
\usepackage[left=1.8cm,top=1cm,right=1.8cm, bottom=1cm,letterpaper, includeheadfoot]{geometry}
\usepackage{framed}
\usepackage{babel}
\usepackage[utf8]{inputenc}
\usepackage{algorithmic}
\usepackage{algorithm}
%\usepackage{enumitem}
\usepackage{enumerate}
\usepackage{multicol}
\usepackage{amssymb, amsmath, amsthm}
\usepackage{subcaption}
\usepackage{graphicx,txfonts}
\usepackage{lmodern,url}
\usepackage{graphicx}
\usepackage{wrapfig}
\usepackage{hyperref}
\usepackage[dvipsnames]{xcolor}
\usepackage{epigraph}
\usepackage{color}
\usepackage{cancel}
\usepackage{tikz}
\def\checkmark{\tikz\fill[scale=0.4](0,.35) -- (.25,0) -- (1,.7) -- (.25,.15) -- cycle;} 
\floatname{algorithm}{Algoritmo}

\makeatletter


\setlength\epigraphwidth{8cm}
\setlength\epigraphrule{0pt}
\usepackage{fancyhdr}
\setlength{\headheight}{15pt} 
\pagestyle{fancy}
\fancypagestyle{plain}{%
    \fancyhf{}
    \lhead{\footnotesize\itshape\bfseries\rightmark}
    \rhead{\footnotesize\itshape\bfseries\leftmark}
    }

\setlength{\parindent}{1cm}
\newenvironment{chapquote}[2][2em]
  {\setlength{\@tempdima}{#1}%
   \def\chapquote@author{#2}%
   \parshape 1 \@tempdima \dimexpr\textwidth-2\@tempdima\relax%
   \itshape}
  {\par\normalfont\hfill--\ \chapquote@author\hspace*{\@tempdima}\par\bigskip}
\makeatother

% macros
\newcommand{\heart}{\ensuremath\heartsuit}
\newcommand{\grad}{\hspace{-2mm}$\phantom{a}^{\circ}$}
\newcommand{\Q}{\mathbb Q}
\newcommand{\R}{\mathbb R}
\newcommand{\N}{\mathbb N}
\newcommand{\Z}{\mathbb Z}
\newcommand{\C}{\mathbb C}
\newcommand{\U}{\mathcal U}
\newcommand{\ssi}{\Longleftrightarrow} %si y solo si
\newcommand{\To}{\Rightarrow}      %implica
\newcommand{\tq}{\mid }            % tal que
\newcommand{\exclusivo}{\veebar }  % o exclusivo
\renewcommand{\vec}[2]{\left(\begin{array}{c}{#1}\\{#2}\end{array}\right)}
\newcommand{\texii}[2]{\begin{minipage}{0.5\textwidth} #1 \end{minipage}  
                     \begin{minipage}{0.5\textwidth} #2 \end{minipage}}

%%%operadores matematicos
\providecommand{\abs}[1]{\lvert#1 \rvert}
\providecommand{\pin}[2]{\left< #1,#2 \right>} %producto interno
\providecommand{\dpartial}[2]{\frac{\partial #1}{\partial #2}} %derivada parcial


%Teoremas, Lemas, etc.
\theoremstyle{plain}
\newtheorem{teo}{Teorema}
\newtheorem{lem}{Lema}
\newtheorem{prop}{Proposici\'on}
\newtheorem{cor}{Corolario}
\newtheorem{prob}{Problema Controlable}
\newtheorem{nota}{Notaci\'on}
\newtheorem{obs}{Observaci\'on}

%%%%%%% inicio documento %%%%%%%
\begin{document}

%============Encabezado estandar============
\newpage
\pagestyle{fancy}
\fancyhf{}
\fancyhead[L]{\textit{Facultad de Ciencias Físicas y Matemáticas}}
\fancyhead[R]{\textit{Universidad de Chile}}

\begin{wrapfigure}{R}{0.2\textwidth} %this figure will be at the right
    \vspace{-5mm}
    \includegraphics[width=0.2\textwidth]{img/fcfm2.png}
\end{wrapfigure}


\noindent
\textbf{MA1101-1 Introducción al Álgebra}\\
\textbf{Profesor: }Leonardo Sánchez C.\\
\textbf{Auxiliar: }Marcelo Navarro

\begin{center}
{\bf \Large Auxiliar 4: Conjuntos II}\\
{12 de Abril de 2018}
\end{center}

\begin{framed}
		\begin{multicols}{2}
		    \begin{itemize}  
		        \item  Se define el conjunto potencia o partes de un conjunto como:
                $\mathcal{P}(A)=2^{A}=\{X:X\subseteq A\}$
                \item Se define el producto cartesiano entre A y B como:
                $A \times B=\{(a,b): a\in A \land b \in B \}$
                \item $\displaystyle x \in \bigcup_{\lambda \in \Lambda} A_{\lambda} \ssi (\exists \lambda \in \Lambda, x \in A_{\lambda})$ 
    
                \item $\displaystyle x \in \bigcap_{\lambda \in \Lambda} A_{\lambda} \ssi (\forall \lambda \in \Lambda, x \in A_{\lambda})$ 
                \vspace{10mm}
                \item $\mathcal{G} \subseteq P(A)$ es partición de $A$ si se cumple:
                    \begin{enumerate}
                        \item $\forall C \in \mathcal{G}, C \neq \emptyset$
                        \item Los elementos de $\mathcal{G}$, son disjuntos de a pares, es decir, $\forall C_{1},C_{2} \in \mathcal{G}$, si $C_{1} \neq C_{2}$, entonces $C_{1}\cap C_{2}= \emptyset$ 
                        \item $\mathcal{G}$ cubre a $A$, es decir, $\bigcup_{C \in \mathcal{G}} C=A$
                        
                    \end{enumerate}
                \item Sea $S\neq \emptyset$, Las particiones triviales de $S$ son:
                    \begin{itemize}
                        \item $\mathcal{G}=\{ S\}$
                        \item $\mathcal{G}=\{ \{x\} : x \in S\}$
                    \end{itemize}
                
                
            \end{itemize}
        \end{multicols}
\end{framed}


\begin{enumerate}[\bf P1.]
    \item \textbf{[Varios]} Sean $A,B,X$ subconjuntos de un universo $\mathcal{U}$. Demuestre que:
        \begin{enumerate}
            \item $(A\cap B)\cup X= A \cap (B\cup X) \iff X \subseteq A$
            \item $(A\cup B)\times X =(A \times X)\cup (B \times X)$
        \end{enumerate}

    \item \textbf{[Conjunto Potencia]} \\
        Sean $A$ y $B$ subconjuntos de un universo $\U$. Demuestre que:
        \begin{enumerate}
            \item $A \cap B = \emptyset \iff P(A)\cap P(B)=\{\emptyset \}$
            \item $[P(A) \cup P(B)=P(A \cup B)] \iff [A\subseteq B \lor B \subseteq A]$.
        \end{enumerate} 
        
    \item \textbf{[Inducción + Conjuntos]}\\
        Demuestre que para todo $n\in \N$ 
            $$\displaystyle \left( \bigcap_{i=1}^n A_i \right)^c=\bigcup_{i=1}^{n}A_{i}^{c}$$
        Donde $A_1,A_2, \dots , A_n$ son conjuntos de un mismo universo $\mathcal{U}$
        
    \item \textbf{[¿Partición de Circunferencias?]} \\ Sea $r \in \mathbb{R}$. Definimos el conjunto  $$ A_r = \{ (x,y) \in \mathbb{R} \times \mathbb{R} \mid x^{2} + y^{2} = r^2 \}$$
        \begin{enumerate}
            \item ¿Es el conjunto $\displaystyle \mathcal{S}= \{ A_r: r\in \R_{+}\cup \{0\} \}$  una partición de $\R\times \R$?
            \item ¿que ocurre si en vez de el conjunto de circunferencias tomamos el conjunto de círculos? es decir si en vez de $A_r$ fuese $C_r=  \{ (x,y) \in \mathbb{R} \times \mathbb{R} \mid x^{2} + y^{2} \leq r^2 \}$
        \end{enumerate}
        
        \item \textbf{[Union de Particiones]} Sea $A$ un conjunto con al menos 3 elementos. Si $P=\{B_1,B_2\}$ es una partición de $A$ y además $C_1$ es una partición de $B_1$, $C_2$ es una partición de $B_2$. Pruebe que $C_1 \cup C_2$ es una partición de $A$.
\end{enumerate}
\end{document}