\documentclass[letterpaper,11pt]{article}

\usepackage[activeacute,spanish]{babel}
\usepackage[left=1.8cm,top=1cm,right=1.8cm, bottom=1cm,letterpaper, includeheadfoot]{geometry}
\usepackage{framed}
\usepackage{babel}
\usepackage[utf8]{inputenc}
\usepackage{algorithmic}
\usepackage{algorithm}
%\usepackage{enumitem}
\usepackage{enumerate}
\usepackage{multicol}
\usepackage{amssymb, amsmath, amsthm}
\usepackage{subcaption}
\usepackage{graphicx,txfonts}
\usepackage{lmodern,url}
\usepackage{graphicx}
\usepackage{wrapfig}
\usepackage{hyperref}
\usepackage[dvipsnames]{xcolor}
\usepackage{epigraph}
\usepackage{color}
\usepackage{cancel}
\usepackage{tikz}
\def\checkmark{\tikz\fill[scale=0.4](0,.35) -- (.25,0) -- (1,.7) -- (.25,.15) -- cycle;} 
\floatname{algorithm}{Algoritmo}

\makeatletter


\setlength\epigraphwidth{8cm}
\setlength\epigraphrule{0pt}
\usepackage{fancyhdr}
\setlength{\headheight}{15pt} 
\pagestyle{fancy}
\fancypagestyle{plain}{%
    \fancyhf{}
    \lhead{\footnotesize\itshape\bfseries\rightmark}
    \rhead{\footnotesize\itshape\bfseries\leftmark}
    }

\setlength{\parindent}{1cm}
\newenvironment{chapquote}[2][2em]
  {\setlength{\@tempdima}{#1}%
   \def\chapquote@author{#2}%
   \parshape 1 \@tempdima \dimexpr\textwidth-2\@tempdima\relax%
   \itshape}
  {\par\normalfont\hfill--\ \chapquote@author\hspace*{\@tempdima}\par\bigskip}
\makeatother

% macros
\newcommand{\heart}{\ensuremath\heartsuit}
\newcommand{\grad}{\hspace{-2mm}$\phantom{a}^{\circ}$}
\newcommand{\Q}{\mathbb Q}
\newcommand{\R}{\mathbb R}
\newcommand{\N}{\mathbb N}
\newcommand{\Z}{\mathbb Z}
\newcommand{\C}{\mathbb C}
\newcommand{\U}{\mathcal U}
\newcommand{\ssi}{\Longleftrightarrow} %si y solo si
\newcommand{\To}{\Rightarrow}      %implica
\newcommand{\tq}{\mid }            % tal que
\newcommand{\exclusivo}{\veebar }  % o exclusivo
\renewcommand{\vec}[2]{\left(\begin{array}{c}{#1}\\{#2}\end{array}\right)}
\newcommand{\texii}[2]{\begin{minipage}{0.5\textwidth} #1 \end{minipage}  
                     \begin{minipage}{0.5\textwidth} #2 \end{minipage}}

%%%operadores matematicos
\providecommand{\abs}[1]{\lvert#1 \rvert}
\providecommand{\pin}[2]{\left< #1,#2 \right>} %producto interno
\providecommand{\dpartial}[2]{\frac{\partial #1}{\partial #2}} %derivada parcial


%Teoremas, Lemas, etc.
\theoremstyle{plain}
\newtheorem{teo}{Teorema}
\newtheorem{lem}{Lema}
\newtheorem{prop}{Proposici\'on}
\newtheorem{cor}{Corolario}
\newtheorem{prob}{Problema Controlable}
\newtheorem{nota}{Notaci\'on}
\newtheorem{obs}{Observaci\'on}

%%%%%%% inicio documento %%%%%%%
\begin{document}

%============Encabezado estandar============
\newpage
\pagestyle{fancy}
\fancyhf{}
\fancyhead[L]{\textit{Facultad de Ciencias Físicas y Matemáticas}}
\fancyhead[R]{\textit{Universidad de Chile}}

\begin{wrapfigure}{R}{0.2\textwidth} %this figure will be at the right
    \vspace{-5mm}
    \includegraphics[width=0.2\textwidth]{img/fcfm2.png}
\end{wrapfigure}


\noindent
\textbf{MA1101-1 Introducción al Álgebra}\\
\textbf{Profesor: }Leonardo Sánchez C.\\
\textbf{Auxiliar: }Marcelo Navarro

\begin{center}
{\bf \Large Auxiliar 5: Funciones I}\\
{23 de Abril de 2018}
\end{center}


\begin{enumerate}[\bf P1.]
    \item \textbf{[Función con Conjuntos]} 
    Sea $E$ un conjunto y $A$ un subconjunto de $E$. Se define la función $f:\mathcal{P}(E) \to \mathcal{P}(E) $ como $f(X)=X\triangle A$ para todo $X \subseteq E$. Probar que $f$ es biyectiva y determine la función inversa de $f$.
    
    
    \item \textbf{[Elegir buena función]} Sea $F$ el conjunto de las funciones de $\R$
    en $\R$. Se define la función $\varphi: F \to \R$ que a cada $f \in F$ le asocia $\varphi(f)=f(0)$. Demuestre que $\varphi$ es una función epiyectiva.
    
    \item \textbf{[Función de funciones]} Considere el conjunto $\mathcal{F}=\{f: \R \to \R ~|~ f \text{ es biyectiva }  \}$, es decir, el conjunto de todas las funciones biyectivas de $\R$ en $\R$. Se define la función $\Psi: \mathcal{F} \times \mathcal{F} \to \mathcal{F}$ dada por
    $$ \Psi(f,g)=(f \circ g)^{-1} $$
    \begin{enumerate}
        \item Demuestre que $\Psi$ esta bien definido $\forall (f,g) \in \mathcal{F} \times \mathcal{F}$
        \item Estudie inyectividad y epiyectividad de $\Psi$
        \item Demuestre que, para todo par $(f,g) \in \mathcal{F} \times \mathcal{F}$,
        $$ \Psi(\Psi(f,g),\Psi(g^{-1},f^{-1}))=id_{\R} $$
    \end{enumerate}
    
    \item \textbf{[Muchas Composiciones]} Sean $A,B$ conjuntos no vacíos y $f: A\to B$,  $g: B\to A$ y $h: B\to B$ funciones tales que:
        \begin{multicols}{3}
        \begin{itemize}
            \item $h$ es biyectiva
            \item $f \circ g=h$
            \item $g \circ f=id_{A}$
        \end{itemize}
        \end{multicols}
    Demuestre que $f$ y $g$ son biyectivas y que $h=id_{B}$
    
    \item \textbf{[Para la Casa]} Sean $f:A\to B$ y $g:A'\to B'$ dos funciones biyectivas.
    Definimos $\mathcal{F}_{A,A'}=\{h: A\to A' ~|~ h \text{ es función }\}$ y $\mathcal{F}_{B,B'}=\{h: B\to B' ~|~ h \text{ es función }\}$.
    Considere 
        \begin{align*}
            \Psi \colon \mathcal{F}_{A,A'} &\to \mathcal{F}_{B,B'}\\
            h &\mapsto \Psi(h)=g\circ h \circ f^{-1}
        \end{align*}
        
        \begin{enumerate}
            \item Probar que $\Psi$ es biyección.
            \item Probar que:
                \begin{enumerate}
                    \item $h$ es inyectiva $\iff$ $\Psi(h)$ es inyectiva.
                    \item $h$ es epiyectiva $\iff$ $\Psi(h)$ es epiyectiva.
                \end{enumerate}
        \end{enumerate}
    
    \item \textbf{[Para la casa 2]} Sean $A,B$ conjuntos no vacíos. Se define $\varphi : A \times B \to A, ~ \varphi(x, y) = x.$
        \begin{enumerate}
            \item Demuestre que $\varphi$ es epiyectiva.
            \item Demuestre que $\varphi$ es biyectiva si y solo si $B$ tiene exactamente un elemento.
            \item Encuentre la inversa de $\varphi$.
        \end{enumerate}
\end{enumerate}

\begin{framed}
		\begin{multicols}{2}
		    \begin{itemize}  
                \item Llamaremos función de $A$ en $B$ a cualquier $f\subseteq A \times B$ tal que:
                        $$(\forall a \in A)(\exists ! b \in B) \hspace{5 mm}(a,b)\in f$$
                    Si $f$ es una función de $A$ en $B$ lo anotaremos como $f:A \to B$, de igual manera si $(a,b)\in f$ lo anotaremos $f(a)=b$.
                \item  Si $f:A\to B$ al conjunto $A$ lo llamaremos dominio de $f$ y a $B$ lo llamaremos el codominio de $f$.
                
                \item Sea $f: A \to B$ y $g: C \to D$ funciones, entonces $f=g$ es equivalente con 
                    \begin{center}
                        $Dom(f)=Dom(g)$\\ 
                        $\land$\\ 
                        $Cod(f)=Cod(g)$\\ 
                        $\land$\\
                        $\forall x \in Dom(f), f(x)=g(x)$
                    \end{center}
                \item Una función $f: A \to B$ es inyectiva $\ssi$ \begin{center}$ \forall x_{1},x_{2} \in A, f(x_{1})=f(x_{2}) \Rightarrow x_{1}=x_{2} $\end{center}
                
                \item Una función $f: A \to B$ es epiyectiva $\ssi$ \begin{center}$\forall y \in B, \exists x \in A, y=f(x) $ \end{center}
    
                \item Una función $f: A \to B$ es biyectiva $\ssi$ es inyectiva y epiyectiva al mismo tiempo $\ssi$ $\forall y \in B, \exists ! x \in A, y=f(x) $
    
                \item Si $f: A \to B$ es biyectiva, entonces existe $f^{-1}:B \to A$ dada por:
                    \begin{center}
                        $ \forall x \in A, \forall y \in B, f^{-1}(y)=x \ssi y=f(x)$
                    \end{center}
    
                \item Si $f: A \to B$ es biyectiva, entonces su inversa $f^{-1}:B \to A$ es tal que:
                    \begin{enumerate}
                        \item $\forall x \in A, f^{-1}(f(x))=x$.
                        \item $\forall y \in B, f(f^{-1}(y))=y$. 
                    \end{enumerate}
                    
                \item Sea $f:A \to B$ y $g:B'\to C$, luego $g \circ f$ tendrá sentido si $B\subseteq B'$ donde $$(g \circ f): A \to C $$
    
                \item $ \forall x \in A, (g \circ f)(x)=g(f(x))$
    
                \item Si $f: A \to B$ es biyectiva, entonces $$f \circ f^{-1}=id_{B} ~\land~ f^{-1} \circ f=id_{A}$$
    
                \item Para $f:A \to B$, las funciones identidades,
                $id_{A} : A\to A,~ id_{B} : B \to B$ actúan como neutros, es decir
                    \begin{itemize}
                        \item $id_{B} \circ f=f$
                        \item $f \circ id_{A}=f$
                     \end{itemize}
    
                \item $\circ$ es asociativa, es decir, $$(h \circ g) \circ f=h \circ (g \circ f)$$ 
    
                \item Se tienen las siguientes propiedades:
                    \begin{enumerate}
                    \item $f$ y $g$ son inyectivas $\Rightarrow g\circ f$ es inyectiva
                    \item $f$ y $g$ son epiyectivas $\Rightarrow g\circ f$ es epiyectiva
                    \item $f$ y $g$ son biyectivas $\Rightarrow g\circ f$ es biyectiva
                    \item $g \circ f$ es inyectiva $\Rightarrow f$ es inyectiva ($g$ no necesariamente)
                    \item $g \circ f$ es epiyectiva $\Rightarrow g$ es epiyectiva ($f$ no necesariamente)
                    \end{enumerate}
        
                \item Si $f$ es biyectiva, entonces $f^{-1}$ tambien es biyectiva y $(f^{-1})^{-1}=f$
    
                \item Sea $f:A \to B$ y $g:B \to A$ si se cumplen dos de las siguientes condiciones, entonces  que $f$ es biyectiva y $f^{-1}=g$.
                    \begin{enumerate}
                        \item $g \circ f=id_A$
                        \item $f \circ g=id_B$
                        \item $g$ es biyectiva
                    \end{enumerate}
                
                \item Si $f$ y $g$ son biyectivas, entonces $$ (g \circ f)^{-1}=f^{-1} \circ g^{-1} $$
                
                
            \end{itemize}
        \end{multicols}
\end{framed}

\end{document}