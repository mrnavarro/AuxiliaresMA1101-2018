\documentclass[letterpaper,11pt]{article}

\usepackage[activeacute,spanish]{babel}
\usepackage[left=1.8cm,top=1cm,right=1.8cm, bottom=1cm,letterpaper, includeheadfoot]{geometry}
\usepackage{framed}
\usepackage{babel}
\usepackage[utf8]{inputenc}
\usepackage{algorithmic}
\usepackage{algorithm}
%\usepackage{enumitem}
\usepackage{enumerate}
\usepackage{multicol}
\usepackage{amssymb, amsmath, amsthm}
\usepackage{subcaption}
\usepackage{graphicx,txfonts}
\usepackage{lmodern,url}
\usepackage{graphicx}
\usepackage{wrapfig}
\usepackage{hyperref}
\usepackage[dvipsnames]{xcolor}
\usepackage{epigraph}
\usepackage{color}
\usepackage{cancel}
\usepackage{tikz}
\def\checkmark{\tikz\fill[scale=0.4](0,.35) -- (.25,0) -- (1,.7) -- (.25,.15) -- cycle;} 
\floatname{algorithm}{Algoritmo}

\makeatletter


\setlength\epigraphwidth{8cm}
\setlength\epigraphrule{0pt}
\usepackage{fancyhdr}
\setlength{\headheight}{15pt} 
\pagestyle{fancy}
\fancypagestyle{plain}{%
    \fancyhf{}
    \lhead{\footnotesize\itshape\bfseries\rightmark}
    \rhead{\footnotesize\itshape\bfseries\leftmark}
    }

\setlength{\parindent}{1cm}
\newenvironment{chapquote}[2][2em]
  {\setlength{\@tempdima}{#1}%
   \def\chapquote@author{#2}%
   \parshape 1 \@tempdima \dimexpr\textwidth-2\@tempdima\relax%
   \itshape}
  {\par\normalfont\hfill--\ \chapquote@author\hspace*{\@tempdima}\par\bigskip}
\makeatother

% macros
\newcommand{\heart}{\ensuremath\heartsuit}
\newcommand{\grad}{\hspace{-2mm}$\phantom{a}^{\circ}$}
\newcommand{\Q}{\mathbb Q}
\newcommand{\R}{\mathbb R}
\newcommand{\N}{\mathbb N}
\newcommand{\Z}{\mathbb Z}
\newcommand{\C}{\mathbb C}
\newcommand{\U}{\mathcal U}
\newcommand{\ssi}{\Longleftrightarrow} %si y solo si
\newcommand{\To}{\Rightarrow}      %implica
\newcommand{\tq}{\mid }            % tal que
\newcommand{\exclusivo}{\veebar }  % o exclusivo
\renewcommand{\vec}[2]{\left(\begin{array}{c}{#1}\\{#2}\end{array}\right)}
\newcommand{\texii}[2]{\begin{minipage}{0.5\textwidth} #1 \end{minipage}  
                     \begin{minipage}{0.5\textwidth} #2 \end{minipage}}

%%%operadores matematicos
\providecommand{\abs}[1]{\lvert#1 \rvert}
\providecommand{\pin}[2]{\left< #1,#2 \right>} %producto interno
\providecommand{\dpartial}[2]{\frac{\partial #1}{\partial #2}} %derivada parcial


%Teoremas, Lemas, etc.
\theoremstyle{plain}
\newtheorem{teo}{Teorema}
\newtheorem{lem}{Lema}
\newtheorem{prop}{Proposici\'on}
\newtheorem{cor}{Corolario}
\newtheorem{prob}{Problema Controlable}
\newtheorem{nota}{Notaci\'on}
\newtheorem{obs}{Observaci\'on}

%%%%%%% inicio documento %%%%%%%
\begin{document}

%============Encabezado estandar============
\newpage
\pagestyle{fancy}
\fancyhf{}
\fancyhead[L]{\textit{Facultad de Ciencias Físicas y Matemáticas}}
\fancyhead[R]{\textit{Universidad de Chile}}

\begin{wrapfigure}{R}{0.2\textwidth} %this figure will be at the right
    \vspace{-5mm}
    \includegraphics[width=0.2\textwidth]{img/fcfm2.png}
\end{wrapfigure}


\noindent
\textbf{MA1101-1 Introducción al Álgebra}\\
\textbf{Profesor: }Leonardo Sánchez C.\\
\textbf{Auxiliar: }Marcelo Navarro

\begin{center}
{\bf \Large Auxiliar 6: Funciones II}\\
{26 de Abril de 2018}
\end{center}

\begin{framed}
		\begin{multicols}{2}
		    \begin{itemize}  
                \item Sea $f:A \to B$ y $A'\subseteq A$ se define el conjunto imagen de $A'$ por $f$ como 
                    \begin{align*}
                        f(A')&=& \{y \in B: \exists x \in     A', f(x)=y\} \\
                        &=&\{f(x) \in B: x \in A'\}\\
                        &=& \bigcup_{x \in A'}\{f(x) \}
                    \end{align*}
    
                \item $f:A \to B$ es epiyectiva $\ssi f(A)=B$
    
                \item Sea $f: A \to B$ y $A_{1},A_{2} \subseteq A$            se tiene que:
                    \begin{enumerate}
                        \item $A_{1} \subseteq A_{2} \Rightarrow f(A_{1})\subseteq f(A_{2})$
                        \item $f(A_{1} \cap A_{2}) \subseteq f(A_{1}) \cap f(A_{2})$
                        \item $f(A_{1} \cup A_{2}) = f(A_{1}) \cup f(A_{2})$
                    \end{enumerate}
    
                \item Sea $f:A \to B$ y $B'\subseteq B$ se define el conjunto imagen de $B'$ por $f$ como 
                    \begin{align*}
                        f^{-1}(B')&=&\{x \in A: f(x) \in B'\}\\
                        &=& \bigcup_{y \in B'}f^{-1}(\{y\}) 
                    \end{align*}
    
                \item \textbf{OBSERVACIÓN}: es muy importante notar la siguiente equivalencia.
                $$x \in f^{-1}(B') \ssi f(x) \in B'$$
                Además se tiene que:
                $$x \in A \Longrightarrow f(x)\in f(A)  $$
    
                \item Sea $f: A \to B$ y $B_{1},B_{2},B' \subseteq B$ se tiene que:
                    \begin{enumerate}
                        \item $B_{1} \subseteq B_{2} \Rightarrow f^{-1}(B_{1})\subseteq f^{-1}(B_{2})$
                        \item $f^{-1}(B_{1} \cap B_{2}) = f^{-1}(B_{1}) \cap f^{-1}(B_{2})$
                        \item $f^{-1}(B_{1} \cup B_{2}) = f^{-1}(B_{1}) \cup f^{-1}(B_{2})$
                        \item $A'\subseteq A \Longrightarrow A'\subseteq f^{-1}(f(A'))$
                        \item $B'\subseteq B \Longrightarrow f(f^{-1}(B')) \subseteq B'$
                    \end{enumerate}
            \end{itemize}
        \end{multicols}
\end{framed}

\begin{enumerate}[\bf P1.]
    \item \textbf{[Elegir un buen número]}\\
        Sea $\mathcal{P}_{F}(\N)=\{A_{k} \subseteq \N : A_{k} \text{ es finito con \textit{k} elementos}, k \in \N \text{ y } A_{k} \neq \emptyset  \}$ y considere $f$ definido por:
        \begin{align*}
            f \colon \mathcal{P}_{F}(\N) &\to \N\\      
            A_{k} &\mapsto f(A_{k})=n_{1}+n_{2}+n_{3}\dots+n_{k}, ~ n_{i}\neq n_{j} ~ \forall i,j
            \end{align*}
        Es decir, es la suma de los $k$ elementos de $A_{k}$. Ejemplo $f(\{0,2,4\})=6=0+2+4$.\\
        Demuestre $f$ que es epiyectiva pero no biyectiva.
    
    \item \textbf{[Calcular conjuntos]} Sea $\U \neq \emptyset$. Sea $f: \mathcal{P}(\U) \times \mathcal{P}(\U) \to \mathcal{P}(\U)$ tal que $f(X,Y)=X\setminus Y$.
        \begin{enumerate}
            \item Determine $f(\{(\U,\emptyset), (\emptyset, A), (A,\emptyset)  \})$. con $A \subseteq \U$
            \item Demuestre que $f^{-1}(\{\U,\emptyset \})=\{( \U,\emptyset )\} \cup \{(X,Y) \in \mathcal{P}(\U) \times \mathcal{P}(\U) ~|~ X\subseteq Y \}$.
            \item Determine $f(D)$ donde $D=\{(X,X) : X \in \mathcal{P}(\U) \}$
        \end{enumerate}
        
    \item \textbf{[Conjunto imagen de diferencia simetrica]} Sea $f:X\to Y$ una función. Pruebe que $\forall A,B \subseteq X$
        $$f(A)\triangle f(B) \subseteq f(A\triangle B)$$
    y muestre que la igualdad se tiene, si $f$ es inyectiva.
    \emph{\textbf{Hint}}: estudie $f(A\setminus B)$ y $f(A)\setminus f(B)$
    
    \item  \textbf{[Varios]} Sea $f:E \to F$ y $g:F \to G$ funciones.
        \begin{enumerate}
            \item Probar que:
                $$ (\forall A,B \subseteq E) ~ [f(A \cap B)=f(A) \cap f(B)] \ssi f \text{ es inyectiva } $$
            \item Sea $A \subseteq G$. Probar que:
                $$ (g \circ f)^{-1}(A)=f^{-1}(g^{-1}(A)) $$
            \item Sea $B \subseteq F$. Probar que:
                $$ f(f^{-1}(B))=B \cap f(E) $$
        \end{enumerate}
        
    \item \textbf{[Si alcanzamos...]} Sea $f:E \to F$ una función:
        \begin{enumerate}
            \item pruebe que $\forall A \subseteq F, f^{-1}(A^{c})=(f^{-1}(A))^{c}$
            \item pruebe que $\forall A \subseteq F,~ \forall B \subseteq F, ~  f^{-1}(A \triangle B)=f^{-1}(A) \triangle f^{-1}(B)$
            \item Sea $A,B \subseteq E$, demuestre que si $ f(B)\setminus f(A) = \emptyset \Longrightarrow f(A \cup B)=f(A) $
            \item Suponga ahora que $f$ es tal que, $\forall A,B \subseteq E,~ [A \subseteq B, A \neq B \Longrightarrow f(A) \neq f(B)]$\\ Demuestre que $f$ es inyectiva.
        \end{enumerate}
        
    \item \textbf{[Para refrescar la memoria]} Sea $f:A\to A$, con $A\neq \emptyset$ una función. Definimos $f^n:A\to A$ por medio de la recurrencia $f^1=f$ y $f^{n+1}=f^n\circ f$. Pruebe $\forall n \in \N$ que:
        \begin{enumerate}
            \item si $f$ es biyectiva, entonces $f^n$ es biyectiva.
            \item si $f$ es biyectiva, entonces $(f^n)^{-1}=(f^{-1})^n$
        \end{enumerate}
    
    \emph{\textbf{Hint}}: inducción.
\end{enumerate}

\end{document}