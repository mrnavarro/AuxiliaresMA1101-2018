\documentclass[letterpaper,10pt]{article}

\usepackage[activeacute,spanish]{babel}
\usepackage[left=1.8cm,top=1cm,right=1.8cm, bottom=1cm,letterpaper, includeheadfoot]{geometry}
\usepackage{framed}
\usepackage{babel}
\usepackage[utf8]{inputenc}
\usepackage{algorithmic}
\usepackage{algorithm}
%\usepackage{enumitem}
\usepackage{enumerate}
\usepackage{multicol}
\usepackage{amssymb, amsmath, amsthm}
\usepackage{subcaption}
\usepackage{graphicx,txfonts}
\usepackage{lmodern,url}
\usepackage{graphicx}
\usepackage{wrapfig}
\usepackage{hyperref}
\usepackage[dvipsnames]{xcolor}
\usepackage{epigraph}
\usepackage{color}
\usepackage{cancel}
\usepackage{tikz}
\def\checkmark{\tikz\fill[scale=0.4](0,.35) -- (.25,0) -- (1,.7) -- (.25,.15) -- cycle;} 
\floatname{algorithm}{Algoritmo}

\makeatletter


\setlength\epigraphwidth{8cm}
\setlength\epigraphrule{0pt}
\usepackage{fancyhdr}
\setlength{\headheight}{15pt} 
\pagestyle{fancy}
\fancypagestyle{plain}{%
    \fancyhf{}
    \lhead{\footnotesize\itshape\bfseries\rightmark}
    \rhead{\footnotesize\itshape\bfseries\leftmark}
    }

\setlength{\parindent}{1cm}
\newenvironment{chapquote}[2][2em]
  {\setlength{\@tempdima}{#1}%
   \def\chapquote@author{#2}%
   \parshape 1 \@tempdima \dimexpr\textwidth-2\@tempdima\relax%
   \itshape}
  {\par\normalfont\hfill--\ \chapquote@author\hspace*{\@tempdima}\par\bigskip}
\makeatother

% macros
\newcommand{\heart}{\ensuremath\heartsuit}
\newcommand{\grad}{\hspace{-2mm}$\phantom{a}^{\circ}$}
\newcommand{\Q}{\mathbb Q}
\newcommand{\R}{\mathbb R}
\newcommand{\N}{\mathbb N}
\newcommand{\Z}{\mathbb Z}
\newcommand{\C}{\mathbb C}
\newcommand{\U}{\mathcal U}
\newcommand{\ssi}{\Longleftrightarrow} %si y solo si
\newcommand{\To}{\Rightarrow}      %implica
\newcommand{\tq}{\mid }            % tal que
\newcommand{\exclusivo}{\veebar }  % o exclusivo
\renewcommand{\vec}[2]{\left(\begin{array}{c}{#1}\\{#2}\end{array}\right)}
\newcommand{\texii}[2]{\begin{minipage}{0.5\textwidth} #1 \end{minipage}  
                     \begin{minipage}{0.5\textwidth} #2 \end{minipage}}

%%%operadores matematicos
\providecommand{\abs}[1]{\lvert#1 \rvert}
\providecommand{\pin}[2]{\left< #1,#2 \right>} %producto interno
\providecommand{\dpartial}[2]{\frac{\partial #1}{\partial #2}} %derivada parcial


%Teoremas, Lemas, etc.
\theoremstyle{plain}
\newtheorem{teo}{Teorema}
\newtheorem{lem}{Lema}
\newtheorem{prop}{Proposici\'on}
\newtheorem{cor}{Corolario}
\newtheorem{prob}{Problema Controlable}
\newtheorem{nota}{Notaci\'on}
\newtheorem{obs}{Observaci\'on}
\newcommand{\cupdot}{\mathbin{\mathaccent\cdot\cup}}
%%%%%%% inicio documento %%%%%%%
\begin{document}

%============Encabezado estandar============
\newpage
\pagestyle{fancy}
\fancyhf{}
\fancyhead[L]{\textit{Facultad de Ciencias Físicas y Matemáticas}}
\fancyhead[R]{\textit{Universidad de Chile}}

\begin{wrapfigure}{R}{0.2\textwidth} %this figure will be at the right
    \vspace{-5mm}
    \includegraphics[width=0.2\textwidth]{img/fcfm2.png}
\end{wrapfigure}


\noindent
\textbf{MA1101-1 Introducción al Álgebra}\\
\textbf{Profesor: }Leonardo Sánchez C.\\
\textbf{Auxiliar: }Marcelo Navarro

\begin{center}
{\bf \Large Auxiliar 13: Estructuras Algebraicas}\\
{02 de agosto de 2018}
\end{center}

\begin{enumerate}[\bf P1.]
    \item Se define $\R^{2}$ la ley de composición interna $*$ por
    $$ (a.b)*(c,d)=(ac,bc+d) $$
    \begin{enumerate}
        \item Estudiar conmutatividad y asociatividad de $*$
        \item Determine el neutro de $(\R^{2},*)$
        \item Determine qué elementos son invertibles para $*$ y calcule sus inversos.
        \item Determine los elementos idempotentes de $(\R^{2},*)$
    \end{enumerate}
    
    \item Sea $(S,\star)$ una estructura algebraica con neutro $e$ y $\star$ una operación asociativa. Para $a\in S$ fijo, invertible para $\star$ y con inverso $a^{-1}\in S$, definimos una nueva operación $\triangle$ en $S$ dada por: $$(\forall x,y\in S)\quad\quad x\triangle y = x\star a \star y.$$
    \begin{enumerate}
	    \item Demuestre que $\triangle$ es una ley de composición interna (en adelante, l.c.i.) asociativa.
	    \item Determine si $\triangle$ tiene neutro, y calcúlelo en caso de existir.
	    \item Caracterice los elementos invertibles para $\triangle$, y calcule el inverso de $a$ con respecto a $\triangle$.
    \end{enumerate}

    
    \item \textbf{[Tablas]}
        \begin{enumerate}
            \item Sea $(S,\ast)$ una estructura algebraica dado por la siguiente tabla:
            \begin{figure}[h]
                \centering
                \begin{tabular}{c|ccc}
                    $\ast$ & $e$ & $a$ & $b$ \\ \hline
                    $e$ & $e$ & $a$ & $b$\\
                    $a$ & $a$ & $e$ & $e$\\
                    $b$ & $b$ & $e$ & $a$
                \end{tabular}
            \end{figure}
            Determine si $\ast$ es asociativa en $S$
            \item Considere $(\Z_{5},\cdot_{5})$
                \begin{enumerate}
                    \item Construya la tabla para la operación $\cdot_{5}$ en $\Z_5$
                    \item Expliqué por que $(\Z_5,\cdot_5)$ no es un grupo.
                    \item Muestre que $(\Z_5 \setminus \{[0]\}, \cdot_5)$ es un grupo abeliano.
                \end{enumerate}
            Recuerdo: un grupo es una estructura algebraica que es asociativo, posee neutro y todo elemento es invertible.
        \end{enumerate}
    
    \item \textbf{[¿Cuantos morfismos?]} 
    
    Sea $m\in \N$ y $\varphi :(\Z_m,+_{m}) \to (\Z,+)$ un homomorfismo cualquiera. Determine el neutro de $\Z_m$ y Demuestre que $\varphi \equiv 0$, es decir, $\varphi$ es la función constante igual a 0.
    
        \item Se define en $\R$ la ley de composición interna $*$ por:
            $$ x*y= \sqrt[5]{x^5+y^5}$$
            Muestre que la biyección $f(x)=x^5$ es un isomorfismo de $(\R,*)$ en $(\R,+)$ y que tambien es un isomorfismo de $(\R,\cdot)$ en $(\R, \cdot)$. Donde las operaciones $+$ y $\cdot$ son la suma y producto usuales en $\R$
    
    
\end{enumerate}

%%%%%%%%%
%Resumen%
%%%%%%%%%
\begin{framed}
	\begin{multicols}{2}
	    \begin{itemize}
	        \item Dado un conjunto $A$ no vacío. Diremos que $*$ es una ley de composición interna (l.c.i) si $*$ es una función  
    \begin{align*}
            *: A\times A & \to A\\
                (x,y)   & \mapsto x*y
    \end{align*}
        
    \item Si $*$ es una l.c.i definida en un conjunto A, entonces al par $(A,*)$ lo llamaremos \textbf{estructura algebraica}.\\ Si sobre $A$ tenemos definida una segunda operación $\triangle$, denotaremos $(A,*,\triangle)$ a la estructura algebraica que considera ambas l.c.i en $A$
    
    \item Sea $(A,*)$ una estructura algebraica.
        \begin{enumerate}
            \item Diremos que $*$ es \textbf{asociativa} si $$\forall x,y,z \in A, (x*y)*z=x*(y*z) $$
            \item Sea $e\in A$. Diremos que $e$ es elemento \textbf{neutro para *} si
            $$\forall x\in A, e*x=x*e=x$$
            \item Si $e\in A$ es el neutro para $*$, diremos que $x\in A$ es \textbf{invertible si:}$$\exists y\in A, y*x=x*y=e$$
            En tal caso, diremos que existe inverso de $x$ y más aun $y$ es un inverso de $x$
            \item Diremos que $*$ es \textbf{conmutativa} si $$ \forall x,y \in A, x*y=y*x $$
            \item Sea $a\in A$. Diremos que $a$ es \textbf{absorbente} si:
            $$\forall x\in A, x*a=a*x=a $$
            \item Sea $a\in A$. Diremos que $a$ es \textbf{idempotente} si $a*a=a$
            \item Sea $a\in A$. Diremos que $a$ es \textbf{cancelable} si $\forall y,z \in A$ se tienen:
            $$a*y=a*z\Rightarrow y=z $$
            $$y*a=z*a\Rightarrow y=z $$
            \item Sea $(A,*,\triangle)$ una estructura algebraica con dos operaciones. Diremos que $\triangle$ \textbf{distribuye} con respecto a $*$ si para todo $x,y,z \in A$, se tienen:
            $$x\triangle(y*z)=(x\triangle y)*(x\triangle z)$$
            $$(y*z)\triangle x=(y\triangle x)*(z \triangle x) $$
        \end{enumerate}
    \item Sea (A,*) una estructura algebraica. Se tiene que $a \in A$ es cancelable si y sólo si las funciones $I_a$ y $D_a$ definidas por $I_a(x)=a*x$ y $D_a(x)=x*a$ para $x \in A$, son inyectivas.
    
    \item En una estructura algebraica $(A,*)$, el elemento neutro es \textbf{único}.
    \item Si la estructura algebraica $(A,*)$ tiene neutro $e$ y $*$ es asociativa, entonces los inversos son unicos.\\
    De esta forma el inverso de $x\in A$, lo podemos denotar sin ambigüedad como $x^{-1}$
    
    \item Sea $(A,*)$ una estructura algebraica asociativa y con neutro $e \in A$ entonces:
        \begin{enumerate}
            \item Si $x\in A$ posee inverso, entonces $x^{-1}$ también. Más aun, $(x^{-1})^{-1}=x$
            \item Si $x,y \in A$ son invertibles entonces $(x*y)^{-1}=y^{-1}*x^{-1}$
            \item Si $x\in A$ posee inverso, entonces x es cancelable.
        \end{enumerate}
    \item Se define $\Z_n=\Z/\equiv_{n}$. A partir de esto podemos definir:
        \begin{itemize}
            \item $[x]_n + [y]_n=[x+y]_n$
            \item $[x]_n \cdot [y]_n=[x\cdot y]_n$
        \end{itemize}
    Recordemos además que si $x\equiv_n y$, entonces $[x]_n=[y]_n$
    
    \item Sea $x_1,x_2,y_1,y_2 \in \Z$ tal que $x_1\equiv_n x_2$ y \\$y_1\equiv_n y_2$. Entonces 
    $$ (x_1+y_1)\equiv_n (x_2+y_2) \text{ y }  (x_1\cdot y_1)\equiv_n (x_2\cdot y_2) $$
    Es decir, si $[x_1]_n = [x_2]_n$ y $[y_1]_n = [y_2]_n$, entonces
    $$ [x_1+y_1]_n=[x_2+y_2]_n \text{ y } [x_1\dot y_1]_n=[x_2 \cdot y_2]_n$$
    
    \item (\textit{Notación}) $x\equiv_ny  \ssi x=y ~(\text{mód } n)$
        \end{itemize}
    \end{multicols}
\end{framed}

\end{document}