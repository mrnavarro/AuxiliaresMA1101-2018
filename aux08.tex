\documentclass[letterpaper,11pt]{article}

\usepackage[activeacute,spanish]{babel}
\usepackage[left=1.8cm,top=1cm,right=1.8cm, bottom=1cm,letterpaper, includeheadfoot]{geometry}
\usepackage{framed}
\usepackage{babel}
\usepackage[utf8]{inputenc}
\usepackage{algorithmic}
\usepackage{algorithm}
%\usepackage{enumitem}
\usepackage{enumerate}
\usepackage{multicol}
\usepackage{amssymb, amsmath, amsthm}
\usepackage{subcaption}
\usepackage{graphicx,txfonts}
\usepackage{lmodern,url}
\usepackage{graphicx}
\usepackage{wrapfig}
\usepackage{hyperref}
\usepackage[dvipsnames]{xcolor}
\usepackage{epigraph}
\usepackage{color}
\usepackage{cancel}
\usepackage{tikz}
\def\checkmark{\tikz\fill[scale=0.4](0,.35) -- (.25,0) -- (1,.7) -- (.25,.15) -- cycle;} 
\floatname{algorithm}{Algoritmo}

\makeatletter


\setlength\epigraphwidth{8cm}
\setlength\epigraphrule{0pt}
\usepackage{fancyhdr}
\setlength{\headheight}{15pt} 
\pagestyle{fancy}
\fancypagestyle{plain}{%
    \fancyhf{}
    \lhead{\footnotesize\itshape\bfseries\rightmark}
    \rhead{\footnotesize\itshape\bfseries\leftmark}
    }

\setlength{\parindent}{1cm}
\newenvironment{chapquote}[2][2em]
  {\setlength{\@tempdima}{#1}%
   \def\chapquote@author{#2}%
   \parshape 1 \@tempdima \dimexpr\textwidth-2\@tempdima\relax%
   \itshape}
  {\par\normalfont\hfill--\ \chapquote@author\hspace*{\@tempdima}\par\bigskip}
\makeatother

% macros
\newcommand{\heart}{\ensuremath\heartsuit}
\newcommand{\grad}{\hspace{-2mm}$\phantom{a}^{\circ}$}
\newcommand{\Q}{\mathbb Q}
\newcommand{\R}{\mathbb R}
\newcommand{\N}{\mathbb N}
\newcommand{\Z}{\mathbb Z}
\newcommand{\C}{\mathbb C}
\newcommand{\U}{\mathcal U}
\newcommand{\ssi}{\Longleftrightarrow} %si y solo si
\newcommand{\To}{\Rightarrow}      %implica
\newcommand{\tq}{\mid }            % tal que
\newcommand{\exclusivo}{\veebar }  % o exclusivo
\renewcommand{\vec}[2]{\left(\begin{array}{c}{#1}\\{#2}\end{array}\right)}
\newcommand{\texii}[2]{\begin{minipage}{0.5\textwidth} #1 \end{minipage}  
                     \begin{minipage}{0.5\textwidth} #2 \end{minipage}}

%%%operadores matematicos
\providecommand{\abs}[1]{\lvert#1 \rvert}
\providecommand{\pin}[2]{\left< #1,#2 \right>} %producto interno
\providecommand{\dpartial}[2]{\frac{\partial #1}{\partial #2}} %derivada parcial


%Teoremas, Lemas, etc.
\theoremstyle{plain}
\newtheorem{teo}{Teorema}
\newtheorem{lem}{Lema}
\newtheorem{prop}{Proposici\'on}
\newtheorem{cor}{Corolario}
\newtheorem{prob}{Problema Controlable}
\newtheorem{nota}{Notaci\'on}
\newtheorem{obs}{Observaci\'on}

%%%%%%% inicio documento %%%%%%%
\begin{document}

%============Encabezado estandar============
\newpage
\pagestyle{fancy}
\fancyhf{}
\fancyhead[L]{\textit{Facultad de Ciencias Físicas y Matemáticas}}
\fancyhead[R]{\textit{Universidad de Chile}}

\begin{wrapfigure}{R}{0.2\textwidth} %this figure will be at the right
    \vspace{-5mm}
    \includegraphics[width=0.2\textwidth]{img/fcfm2.png}
\end{wrapfigure}


\noindent
\textbf{MA1101-1 Introducción al Álgebra}\\
\textbf{Profesor: }Leonardo Sánchez C.\\
\textbf{Auxiliar: }Marcelo Navarro

\begin{center}
{\bf \Large Auxiliar 8: Sumatorias}\\
{07 de Mayo de 2018}
\end{center}

\begin{framed}
	\begin{multicols}{2}
	    \begin{itemize}  
            \item $\displaystyle \sum_{i=p}^{q}a_{i}=a_p+a_{(p+1)}+a_{(p+2)}+\cdots+a_{(q-1)}+a_q$
            \item $\displaystyle \sum_{k=m}^{n}1=n-m+1$
            \item $\displaystyle \sum_{k=m}^{n} \lambda\cdot a_{k}= \lambda\cdot \sum_{k=m}^{n}  a_{k}$
            \item $\displaystyle \sum_{k=m}^{n} a_{k} \pm b_{k}= \sum_{k=m}^{n} a_{k} \pm \sum_{k=m}^{n} b_{k}$
            \item \textbf{ Traslación de índices }$$\displaystyle \sum_{k=m}^{n} a_k = \sum_{k=m-s}^{n-s} a_{k+s} = \sum_{k=m+s}^{n+s} a_{k-s}$$
            \item $\displaystyle \sum_{k=m}^{n} a_k= \sum_{k=m}^{s} a_k + \sum_{k=s+1}^{n} a_k ~~~ \text{para } m \leq s < n$\\
            \item \textbf{telescópica: } $$\displaystyle \sum_{k=m}^{n} (a_k - a_{k+1})= a_m - a_{n+1}  $$
            \item $\displaystyle \sum_{k=0}^{n} k= \frac{n(n+1)}{2}$
            \item $\displaystyle \sum_{k=0}^{n} k^2= \frac{n(n+1)(2n+1)}{6}  $
            \item $\displaystyle \sum_{k=0}^{n} k^3= (\frac{n(n+1)}{2})^{2} $\\
            \textbf{obs: } para las 3 sumas anteriores, si parten de 0 o de 1 es la misma formula
            \item \textbf{Geometrica: }
            $$\displaystyle \sum_{k=0}^{n} r^k= \frac{r^{n+1}-1}{r-1} $$
            para $r\neq 1$
        \end{itemize}
    \end{multicols}
\end{framed}

\begin{enumerate}[\bf P1.]
    \item Calcule
        \begin{multicols}{2}
            \begin{enumerate}
                \item $\displaystyle\sum_{k=3}^{n-1} (k-2)(k+1)$
                \item $\displaystyle \sum_{i=\frac{m(m-1)}{2}+1}^{\frac{m(m+1)}{2}+1}(2i-1) $
                \item $\displaystyle \sum_{k=1}^{n} ln(1+\frac{1}{k})$
                \item $\displaystyle\sum_{k=3}^n b_{k}-b_{k-3}$
                \item $\displaystyle\sum\limits_{k=1}^n \frac{1}{\sqrt{k}(k+1) + k\sqrt{k+1}}$
                \item $\displaystyle\sum_{k=0}^n{ kk! }$
                \item $\displaystyle \sum_{k=1}^n \frac{nk2^k}{(k+2)!}$
            \end{enumerate}
        \end{multicols}
    
    \item Escriba la siguiente suma en notación de sumatoria y luego calcule su valor
    
    $$\dfrac{1}{n(n+1)}+ \dfrac{1}{(n+1)(n+2)}+ \dots + \dfrac{1}{(2n-1)2n} $$
    
    \item Definamos para $n\geq 1, r\neq 1$.
    $$\displaystyle S_{n}=\sum_{k=1}^{n}kr^k $$
    \begin{enumerate}
        \item Demuestre sin usar inducción que:
        $$\displaystyle S_{n}= r(S_{n}-nr^n)+\sum_{k=0}^{n-1}r^{k+1} $$
        \item Demuestre sin usar inducción que:
        $$\displaystyle S_{n}= \frac{r-(n+1)r^{n+1}+nr^{n+2}}{(1-r)^2} $$
    \end{enumerate}
    
    \item Se definen los números armónicos como:
    $$\displaystyle H_n=\sum_{k=1}^{n} \frac{1}{k}=1+\frac{1}{2}+\frac{1}{3}+\cdots + \frac{1}{n}$$
    Demuestre por inducción que 
    $$\displaystyle 1+\frac{n}{2} \leq H_{2^{n}} \leq 1+n $$
    
\end{enumerate}

\end{document}