\documentclass[letterpaper,11pt]{article}

\usepackage[activeacute,spanish]{babel}
\usepackage[left=1.8cm,top=1cm,right=1.8cm, bottom=1cm,letterpaper, includeheadfoot]{geometry}
\usepackage{framed}
\usepackage{babel}
\usepackage[utf8]{inputenc}
\usepackage{algorithmic}
\usepackage{algorithm}
%\usepackage{enumitem}
\usepackage{enumerate}
\usepackage{multicol}
\usepackage{amssymb, amsmath, amsthm}
\usepackage{subcaption}
\usepackage{graphicx,txfonts}
\usepackage{lmodern,url}
\usepackage{graphicx}
\usepackage{wrapfig}
\usepackage{hyperref}
\usepackage[dvipsnames]{xcolor}
\usepackage{epigraph}
\usepackage{color}
\usepackage{cancel}
\usepackage{tikz}
\def\checkmark{\tikz\fill[scale=0.4](0,.35) -- (.25,0) -- (1,.7) -- (.25,.15) -- cycle;} 
\floatname{algorithm}{Algoritmo}

\makeatletter


\setlength\epigraphwidth{8cm}
\setlength\epigraphrule{0pt}
\usepackage{fancyhdr}
\setlength{\headheight}{15pt} 
\pagestyle{fancy}
\fancypagestyle{plain}{%
    \fancyhf{}
    \lhead{\footnotesize\itshape\bfseries\rightmark}
    \rhead{\footnotesize\itshape\bfseries\leftmark}
    }

\setlength{\parindent}{1cm}
\newenvironment{chapquote}[2][2em]
  {\setlength{\@tempdima}{#1}%
   \def\chapquote@author{#2}%
   \parshape 1 \@tempdima \dimexpr\textwidth-2\@tempdima\relax%
   \itshape}
  {\par\normalfont\hfill--\ \chapquote@author\hspace*{\@tempdima}\par\bigskip}
\makeatother

% macros
\newcommand{\heart}{\ensuremath\heartsuit}
\newcommand{\grad}{\hspace{-2mm}$\phantom{a}^{\circ}$}
\newcommand{\Q}{\mathbb Q}
\newcommand{\R}{\mathbb R}
\newcommand{\N}{\mathbb N}
\newcommand{\Z}{\mathbb Z}
\newcommand{\C}{\mathbb C}
\newcommand{\U}{\mathcal U}
\newcommand{\ssi}{\Longleftrightarrow} %si y solo si
\newcommand{\To}{\Rightarrow}      %implica
\newcommand{\tq}{\mid }            % tal que
\newcommand{\exclusivo}{\veebar }  % o exclusivo
\renewcommand{\vec}[2]{\left(\begin{array}{c}{#1}\\{#2}\end{array}\right)}
\newcommand{\texii}[2]{\begin{minipage}{0.5\textwidth} #1 \end{minipage}  
                     \begin{minipage}{0.5\textwidth} #2 \end{minipage}}

%%%operadores matematicos
\providecommand{\abs}[1]{\lvert#1 \rvert}
\providecommand{\pin}[2]{\left< #1,#2 \right>} %producto interno
\providecommand{\dpartial}[2]{\frac{\partial #1}{\partial #2}} %derivada parcial


%Teoremas, Lemas, etc.
\theoremstyle{plain}
\newtheorem{teo}{Teorema}
\newtheorem{lem}{Lema}
\newtheorem{prop}{Proposici\'on}
\newtheorem{cor}{Corolario}
\newtheorem{prob}{Problema Controlable}
\newtheorem{nota}{Notaci\'on}
\newtheorem{obs}{Observaci\'on}

%%%%%%% inicio documento %%%%%%%
\begin{document}

%============Encabezado estandar============
\newpage
\pagestyle{fancy}
\fancyhf{}
\fancyhead[L]{\textit{Facultad de Ciencias Físicas y Matemáticas}}
\fancyhead[R]{\textit{Universidad de Chile}}

\begin{wrapfigure}{R}{0.2\textwidth} %this figure will be at the right
    \vspace{-5mm}
    \includegraphics[width=0.2\textwidth]{img/fcfm2.png}
\end{wrapfigure}


\noindent
\textbf{MA1101-1 Introducción al Álgebra}\\
\textbf{Profesor: }Leonardo Sánchez C.\\
\textbf{Auxiliar: }Marcelo Navarro

\begin{center}
{\bf \Large Auxiliar 11: Binomio de Newton y Conjuntos finitos}\\
{\today}
\end{center}

\begin{enumerate}[\bf P1.]
    
    \item \textbf{[¿Que es un coeficiente binomial?]}
    
    Sean $m,k,n$ naturales tal que $m\leq k \leq n$. Pruebe, usando argumentos combinatoriales que:
                $$\binom{n}{k}\binom{k}{m}=\binom{n}{m}\binom{n-m}{k-m} ~~\text{   y   }~~ \binom{n}{k}=\binom{n-1}{k} + \binom{n-1}{k-1}$$
    \item \textbf{[Binomio de newton]}
    \begin{enumerate}
        \item Demuestre que
            $$\displaystyle \sum_{m=0}^{n} \sum_{k=m}^{n} \binom{n}{k}\binom{k}{m} = 3^n$$
        \item Pruebe sin usar inducción que para todo $n\in \N$, se tiene que
            $$\sum_{k=0}^n \binom{n}{k} \dfrac{(-1)^k}{k+1} = \dfrac{1}{n+1}$$
    \end{enumerate}
    
    
    \item \textbf{[¿Donde queda $k$?]}
    
    Determine el valor de $k$ si los coeficientes de $x^k$ y de $x^{k+1}$ en el desarrollo de $(3x+2)^{14}$ son iguales.
    
    \item \textbf{[Nuestra primera biyección]}
    
    Pruebe que $|A\times B|=|B\times A|$ con $A$ y $B$ finitos. ¿Es necesario que sean $A$ y $B$ finitos en su demostración?
    
    \item \textbf{[Principio del Palomar]}
    
    Sea $A=[0..n]$ y considere la secuencia $(x_0,x_1,\dots)$ de elementos en $A$. Pruebe que existen $\ell,j\in \N$ tales que $x_\ell=x_j$.
    
    \item \textbf{[Control 2017 - C4]}
    
    Demuestre que $|\{ 2i+1 : i\in \N, 0\leq i < 2^{n-1}, n \in \{1,\dots,m\}\} |=2^{m-1}$
    
    \begin{center}
        Propuestos
    \end{center}
    
    \item Demuestre el teorema del binomio de newton usando argumentos combinatoriales. Para esto, compare $(x+y)^n=\underbrace{(x+y)(x+y) \cdots (x+y)}_{n-veces}$ con la expansión de $\sum_{k=0}^{n}\binom{n}{k}x^{k}y^{n-k}$ y piense en cuantos elementos de la forma $x^{n-j} y^j$ hay para un $j\in [0..n]$.
    
    \item Determine $|A\cup B \cup C|$ en función del cardinal de $A$,$B$ y $C$, y sus intersecciones.
    
\end{enumerate}

\end{document}