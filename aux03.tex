\documentclass[letterpaper,11pt]{article}

\usepackage[activeacute,spanish]{babel}
\usepackage[left=1.8cm,top=1cm,right=1.8cm, bottom=1cm,letterpaper, includeheadfoot]{geometry}
\usepackage{framed}
\usepackage{babel}
\usepackage[utf8]{inputenc}
\usepackage{algorithmic}
\usepackage{algorithm}
%\usepackage{enumitem}
\usepackage{enumerate}
\usepackage{multicol}
\usepackage{amssymb, amsmath, amsthm}
\usepackage{subcaption}
\usepackage{graphicx,txfonts}
\usepackage{lmodern,url}
\usepackage{graphicx}
\usepackage{wrapfig}
\usepackage{hyperref}
\usepackage[dvipsnames]{xcolor}
\usepackage{epigraph}
\usepackage{color}
\usepackage{cancel}
\usepackage{tikz}
\def\checkmark{\tikz\fill[scale=0.4](0,.35) -- (.25,0) -- (1,.7) -- (.25,.15) -- cycle;} 
\floatname{algorithm}{Algoritmo}

\makeatletter


\setlength\epigraphwidth{8cm}
\setlength\epigraphrule{0pt}
\usepackage{fancyhdr}
\setlength{\headheight}{15pt} 
\pagestyle{fancy}
\fancypagestyle{plain}{%
    \fancyhf{}
    \lhead{\footnotesize\itshape\bfseries\rightmark}
    \rhead{\footnotesize\itshape\bfseries\leftmark}
    }

\setlength{\parindent}{1cm}
\newenvironment{chapquote}[2][2em]
  {\setlength{\@tempdima}{#1}%
   \def\chapquote@author{#2}%
   \parshape 1 \@tempdima \dimexpr\textwidth-2\@tempdima\relax%
   \itshape}
  {\par\normalfont\hfill--\ \chapquote@author\hspace*{\@tempdima}\par\bigskip}
\makeatother

% macros
\newcommand{\heart}{\ensuremath\heartsuit}
\newcommand{\grad}{\hspace{-2mm}$\phantom{a}^{\circ}$}
\newcommand{\Q}{\mathbb Q}
\newcommand{\R}{\mathbb R}
\newcommand{\N}{\mathbb N}
\newcommand{\Z}{\mathbb Z}
\newcommand{\C}{\mathbb C}
\newcommand{\U}{\mathcal U}
\newcommand{\ssi}{\Longleftrightarrow} %si y solo si
\newcommand{\To}{\Rightarrow}      %implica
\newcommand{\tq}{\mid }            % tal que
\newcommand{\exclusivo}{\veebar }  % o exclusivo
\renewcommand{\vec}[2]{\left(\begin{array}{c}{#1}\\{#2}\end{array}\right)}
\newcommand{\texii}[2]{\begin{minipage}{0.5\textwidth} #1 \end{minipage}  
                     \begin{minipage}{0.5\textwidth} #2 \end{minipage}}

%%%operadores matematicos
\providecommand{\abs}[1]{\lvert#1 \rvert}
\providecommand{\pin}[2]{\left< #1,#2 \right>} %producto interno
\providecommand{\dpartial}[2]{\frac{\partial #1}{\partial #2}} %derivada parcial


%Teoremas, Lemas, etc.
\theoremstyle{plain}
\newtheorem{teo}{Teorema}
\newtheorem{lem}{Lema}
\newtheorem{prop}{Proposici\'on}
\newtheorem{cor}{Corolario}
\newtheorem{prob}{Problema Controlable}
\newtheorem{nota}{Notaci\'on}
\newtheorem{obs}{Observaci\'on}

%%%%%%% inicio documento %%%%%%%
\begin{document}

%============Encabezado estandar============
\newpage
\pagestyle{fancy}
\fancyhf{}
\fancyhead[L]{\textit{Facultad de Ciencias Físicas y Matemáticas}}
\fancyhead[R]{\textit{Universidad de Chile}}

\begin{wrapfigure}{R}{0.2\textwidth} %this figure will be at the right
    \vspace{-5mm}
    \includegraphics[width=0.2\textwidth]{img/fcfm2.png}
\end{wrapfigure}


\noindent
\textbf{MA1101-1 Introducción al Álgebra}\\
\textbf{Profesor: }Leonardo Sánchez C.\\
\textbf{Auxiliar: }Marcelo Navarro

\begin{center}
{\bf \Large Auxiliar 3: Conjuntos}\\
{05 de Abril de 2018}
\end{center}

\begin{framed}
		\begin{multicols}{2}
			\begin{itemize}
                \item $x \in A$ se lee \textit{``x pertenece a A"}.
                \item $A \subseteq B$ se lee \textit{``A es             subconjunto de B"}.
                \item El cjto universo se simboliza por lo general             con $\mathcal{U}$ y el vacio con $\emptyset$.
                \item $x \in \emptyset$ siempre será falso.
                \item Escribir un conjunto que sea solo de los         elementos que cumplen una proposición             \textit{p(x)}.\\ $A=\{x \in \mathcal{U}: p(x) \}$.
                \item $A \subseteq B \ssi (\forall x \in \U)[ x \in          A \Rightarrow x \in B$].
                \item $A = B \ssi (\forall x \in \U)[ x \in A           \Leftrightarrow x \in B$].\\
                    O en terminos de subconjuntos:\\
                        $A = B \ssi A \subseteq B \land B \subseteq A$
                        
                \item Prop: \begin{enumerate}
                                \item $A=A$
                                \item $A=B  \ssi B=A$
                                \item $A=B  \land B=C \ssi A=C$
                                \item $A \subseteq A$
                                \item $A \subseteq B \land B \subseteq C \Longrightarrow A \subseteq C$
                            \end{enumerate}
                            
                \item Sean $A$ y $B$ dos subconjuntos de $\U$. Se definen las siguientes operaciones entre conjuntos:
                    \begin{enumerate}
                        \item $A \cup B=\{x \in \U: x \in A \lor x \in B \}$
                        \item $A \cap B=\{x \in \U: x \in A \land x \in B \}$
                        \item $A^{c}=\{x \in \U: x \not \in A \}$
                        \item $A \setminus B =\{x \in \U: x \in A \land x \not \in B \}=A\cap B^{c}$
                        \item $A\triangle B= (A \setminus B) \cup (B \setminus A) = (A \cup B) \setminus (A\cap B)$
                    \end{enumerate}
                    
                \item La unión, y intersección cumplen conmutatividad, asociatividad, distributividad, De Morgan.
                \item  Se define el conjunto potencia o partes de un conjunto como:
                $\mathcal{P}(A)=2^{A}=\{X:X\subseteq A\}$
                \item Se define el producto cartesiano entre A y B como:
                $A \times B=\{(a,b): a\in A \land b \in B \}$
            \end{itemize}	
        \end{multicols}
\end{framed}

\begin{enumerate}[\bf P1.]
    \item \textbf{[Calentamiento]}\\
    Determine si las siguientes afirmaciones son verdaderas o falsas. Justifique brevemente su respuesta.
        \begin{multicols}{3}
            \begin{enumerate}
                \item $\emptyset \subseteq \emptyset$
    	        \item $\emptyset \subseteq  \{ \emptyset \}$
    	        \item $\emptyset  \in \{ \emptyset \}$
    	        \item $ x \in \{x\}$
    	        \item $ x \in \{\{x\}\}$
    	        \item $ \{a,b\} \subseteq \{a,b,c, \{a,b,c \} \} $
    	        \item $ \{a,b\} \in \{a,b,c, \{a,b,c \} \} $
    	        \item $ \{a,\emptyset \} \in \{a,\{a, \emptyset \} \}$
    	        \item $ \{a,\emptyset \} \subseteq  \{a,\{a, \emptyset \} \}$
            \end{enumerate}
        \end{multicols} 
    
    \item \textbf{[Igualdades de Conjuntos]}
        \begin{enumerate}
            \item $(A \cap B) \cup (A^{c} \cap B) \cup (A^{c} \cap B^{c}) = A^{c}\cup B$
            \item si $A,B$ no son vacios, entonces $ A\cap B = \emptyset \Longrightarrow A \cup B^{c}=B^{c}$
            \item $P(A\cap B)=P(A)\cap P(B)$
            \item demostrar que $\forall X,Y \subseteq \U$ se tiene la siguiente propiedad:
	    $$
	    (X \cup A = Y \cup A) \land (X \cap A = Y \cap A) \implies X=Y
	    $$
	    donde $A \subseteq \U$ es un conjunto fijo.
        \end{enumerate}
        
    \item \textbf{[Diferencia Simétrica]} \\
    Sean $A,B,C \in P(\U)$.
    \begin{enumerate}
	    \item Demuestre la \emph{propiedad cancelativa} de la diferencia simetrica, es decir:
	    $$
	    A \bigtriangleup B = A \bigtriangleup C \implies B=C
	    $$
	    \item Use lo anterior para demostrar que: $$B=(A\cap B^c) \cup (A^c \cap B) \iff A = \emptyset$$
	    \item Use la propiedad cancelativa para demostrar la pregunta $P2.d)$
	   \end{enumerate}
        
    \item \textbf{[Ecuación con conjuntos]}\\
        Sean $A,B$ conjuntos no vacíos Demuestre que
            \begin{enumerate}
                \item $A\cap B = \emptyset \ssi (A\cup B)\setminus B=A$
                \item Encuentre un conjunto $X$ que verifique las siguientes ecuaciones:
                $$ A\cup X= A\cup B  ~~ \text{y}~~ A\cap X=\emptyset$$
                \item Demuestre que la solución que encontró es única.
            \end{enumerate}
        
    \item \textbf{[\heart\ Familia de Conjuntos \heart]}\\
    Sea $\odot$ una ley de operación entre conjuntos definida por $A\odot B=A^{C}\cap B^{C}$. Considere un universo $\U$ y $\mathcal{F}$ $\subseteq \mathcal{P}(\U)$ un conjunto no vacío, al cual llamaremos ``\textit{familia}". En esta pregunta, La \textit{familia} $\mathcal{F}$ cumple la siguiente propiedad $\forall A,B \in \mathcal{F}, A\odot B \in \mathcal{F}$.\\
    Sea $A,B \in \mathcal{F}$ demuestre que:

    \begin{enumerate}
        \item $A^{C} \in \mathcal{F}$
        \item $A\cap B \in \mathcal{F}$
        \item $A\cup B \in \mathcal{F}$
        \item $A\triangle B \in \mathcal{F}$
        \item $\emptyset \in \mathcal{F} \land \U \in \mathcal{F}$
    \end{enumerate}    



\end{enumerate}
\end{document}